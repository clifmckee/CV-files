% Written by Ava Hoffman
% Please use as you like, but it would be nice if you credited me :)
% 
% Use LaTeX to compile and ensure style file (cv.cls) is in the same directory.
%
%%%%%%%%%%%%%%%%%%%%%%%%%%%%%%%%%%%%%%%%%

%----------------------------------------------------------------------------------------
%	PREAMBLE
%----------------------------------------------------------------------------------------

\documentclass{cv}

\begin{document}

%----------------------------------------------------------------------------------------
%	PROFESSIONAL DATA
%----------------------------------------------------------------------------------------

\name{CLIFTON D. MCKEE}

\section*{PROFESSIONAL DATA}

\subsection*{Contact Information}

Johns Hopkins University \\
Bloomberg School of Public Health \\
Department of Epidemiology \\
615 N. Wolfe St., Room E6008\\
Baltimore, MD 21205 \\
Email: \emailcontact{cmckee7@jhu.edu} $\cdot$ \emailcontact{clifton.mckee@gmail.com} \\
Pronouns: he/him/his

\subsection*{Website \& Social Media}

Website: \href{https://clifmckee.github.io/}{clifmckee.github.io} \\
GitHub: \href{https://github.com/clifmckee/}{clifmckee} \\
Twitter: \href{https://twitter.com/clifmckee/}{clifmckee}

%----------------------------------------------------------------------------------------
%	EDUCATION AND TRAINING
%----------------------------------------------------------------------------------------

\section*{EDUCATION AND TRAINING}

\subsection*{Degrees}

Ph.D. / 2020 $\cdot$ Ecology $\cdot$ Colorado State University $\cdot$ Fort Collins, CO \\
Thesis: Evolutionary and ecological processes in microparasite communities of bats \\
Supervisor: Prof. Colleen Webb

M.S. / 2015 $\cdot$ Ecology $\cdot$ Colorado State University $\cdot$ Fort Collins, CO \\
Thesis: Spatial, demographic, and phylogenetic patterns of \textit{Bartonella} diversity in bats \\
Supervisor: Prof. Colleen Webb

B.S. / 2011 $\cdot$ Ecology and Evolution $\cdot$ University of Pittsburgh $\cdot$ Pittsburgh, PA \\
B.A. / 2011 $\cdot$ Environmental Studies $\cdot$ University of Pittsburgh $\cdot$ Pittsburgh, PA \\
\textit{magna cum laude}

\subsection*{Postdoctoral Training}

2020--2022 $\cdot$ Johns Hopkins Bloomberg School of Public Health $\cdot$ Baltimore, MD \\
Member: \href{http://www.iddynamics.jhsph.edu/}{Infectious Disease Dynamics} $\cdot$ \href{https://batonehealth.org/}{Bat One Health} \\
Supervisor: Prof. Emily Gurley

%----------------------------------------------------------------------------------------
%	PROFESSIONAL EXPERIENCE
%----------------------------------------------------------------------------------------

\section*{PROFESSIONAL EXPERIENCE}

\subsection*{Johns Hopkins Bloomberg School of Public Health}

Research Associate $\cdot$ Department of Epidemiology (January 2023 -- present)

Postdoctoral Fellow $\cdot$ Department of Epidemiology (January 2020 -- December 2022)

\subsection*{Other Professional Experience}

Research Assistant $\cdot$ Colorado State University $\cdot$ Fort Collins, CO (2015 \& 2019) \\
Department of Biology $\cdot$ Supervisor: Prof. Colleen Webb \\
\textit{Data analysis and modeling of livestock movement and pathogen spread}

Guest Researcher $\cdot$ CDC Division of Vector-Borne Diseases $\cdot$ Fort Collins, CO (2014--2019) \\
Bacterial Diseases Branch $\cdot$ Supervisor: Dr. Michael Kosoy \\
\textit{Molecular detection and phylogenetic analysis of bacterial pathogens in wildlife}

Regular Fellow $\cdot$ CDC Division of Vector-Borne Diseases $\cdot$ Fort Collins, CO (2015--2017) \\
Bacterial Diseases Branch $\cdot$ Supervisor: Dr. Michael Kosoy \\
\textit{Experimental infection of rodents and fleas with bacterial pathogens}

Field Assistant \& Field Research Specialist $\cdot$ Emory University $\cdot$ Atlanta, GA (2010 \& 2013) \\
Department of Psychology $\cdot$ Supervisors: Prof. Donna Maney, Prof. Brent Horton \\
\textit{Behavioral assays and tissue sampling of wild birds in Maine}

Laboratory Technician $\cdot$ W.L. Gore \& Associates $\cdot$ Newark, DE (2011--2013) \\
Industrial Products Division, New Product Development \\
\textit{Prototype development and performance testing for new filtration products}

Field Technician $\cdot$ University of California, Santa Cruz $\cdot$ Santa Cruz, CA (2011) \\
Department of Ecology and Evolutionary Biology $\cdot$ Supervisor: Prof. Marm Kilpatrick \\
\textit{Blood sampling from wild birds for West Nile virus surveillance in Washington, DC area}

%----------------------------------------------------------------------------------------
%	PROFESSIONAL ACTIVITIES
%----------------------------------------------------------------------------------------

\section*{PROFESSIONAL ACTIVITIES}

\subsection*{Society Memberships and Leadership}

Member $\cdot$ North American Society for Bat Research (2021--present)

Member $\cdot$ American Society of Tropical Medicine and Hygiene (2021 \& 2023)

Member $\cdot$ Society for the Study of Evolution (2019 \& 2021)

Member $\cdot$ American Society of Naturalists (2019)

Member \& Volunteer $\cdot$ Ecological Society of America (2014--2019)

Volunteer \& Vice President $\cdot$ Front Range Student Ecology Symposium (2014--2019)

\subsection*{Community Service}

Exam Writer $\cdot$ Disease Detectives, Maryland Science Olympiad (2021--2022)

Moderator $\cdot$ Rocky Mountain Regional Middle School Science Bowl (2018 \& 2019)

Judge $\cdot$ O'Dea Elementary School Science Fair (2015)

%----------------------------------------------------------------------------------------
%	EDITORIAL AND OTHER PEER REVIEW ACTIVITIES
%----------------------------------------------------------------------------------------

\section*{EDITORIAL AND OTHER PEER REVIEW ACTIVITIES}

\textbf{Peer Review}: Acta Chiropterologica $\cdot$ Acta Tropica $\cdot$ BMJ $\cdot$ Current Zoology $\cdot$ Frontiers in Microbiology $\cdot$ Frontiers in Veterinary Science $\cdot$ Infection, Genetics and Evolution $\cdot$ Journal of Animal Ecology $\cdot$ Journal of Medical Entomology $\cdot$ Journal of Zoological Systematics and Evolutionary Research $\cdot$ Microbial Ecology $\cdot$ Microbial Pathogenesis $\cdot$ Molecular Ecology $\cdot$ Nature Microbiology $\cdot$ Pathogens and Global Health $\cdot$ PLOS Neglected Tropical Diseases $\cdot$ PLOS ONE $\cdot$ The European Zoological Journal $\cdot$ The Lancet Planetary Health $\cdot$ The Science of Nature $\cdot$ Scientific Reports $\cdot$ Systematic Biology $\cdot$ Tropical Biomedicine $\cdot$ Veterinary Microbiology $\cdot$ Viruses $\cdot$ Zoonoses and Public Health

%----------------------------------------------------------------------------------------
%	HONORS AND AWARDS
%----------------------------------------------------------------------------------------

\section*{HONORS AND AWARDS}

Best Poster Award $\cdot$ Ecology and Evolution of Infectious Disease Conference (2019)

Second Place Poster Award $\cdot$ Front Range Student Ecology Symposium (2019)

Vice President for Research Fellowship $\cdot$ Colorado State University (2018) % $4000

Department of Biology Travel Awards $\cdot$ Colorado State University (2014 \& 2017) % $2000

Sharon E. and David E. Kabes Scholarship $\cdot$ Colorado State University (2015) % 1150

Graduate Degree Program in Ecology Research Grant $\cdot$ Colorado State University (2014) % $2000

Department of Biology Graduate Fellowship $\cdot$ Colorado State University (2013) % $1000

Phi Beta Kappa $\cdot$ University of Pittsburgh (2011)

Honors College Scholarship $\cdot$ University of Pittsburgh (2007) % $4000

%----------------------------------------------------------------------------------------
%	PUBLICATIONS
%----------------------------------------------------------------------------------------

\section*{PUBLICATIONS}

\subsection*{Journal Articles (Peer Reviewed)}

{\small \textsuperscript{*}Corresponding author \textsuperscript{\dag}Equal contribution \textsuperscript{\ddag}Student advisee}

\begin{pubenum}

\item Fagre AC, Islam A, Reeves WK, Kading RC, Plowright RK, Gurley ES, \textbf{McKee CD}\textsuperscript{*}. \textit{Bartonella} infection in fruit bats and bat flies, Bangladesh. \textit{Microbial Ecology}. 2023. DOI: \href{https://doi.org/10.1007/s00248-023-02293-9}{10.1007/s00248-023-02293-9}

\item Szentiványi T, \textbf{McKee C}, Jones G, Foster JT. Trends in bacterial pathogens of bats: global distribution and knowledge gaps. \textit{Transboundary and Emerging Diseases}. 2023; 9285855. DOI: \href{https://doi.org/10.1155/2023/9285855}{10.1155/2023/9285855}

\item Seidlova V, Straková P, Kejíková R, Nemcova M, Bartonička T, Salát J, Dufková L, Šikutová S, Mendel J, \textbf{McKee C}, Zukal J, Pikula J, Rudolf I. Detection of \textit{Leptospira} species in bat cadavers, Czech and Slovak Republics. \textit{Emerging Microbes \& Infections}. 2022; 11(1): 2211--2213. DOI: \href{https://doi.org/10.1080/22221751.2022.2117095}{10.1080/22221751.2022.2117095}

\item Kejíková R, \textbf{McKee C}, Straková P, Šikutová S, Mendel J, Rudolf I. First detection of \textit{Bartonella} spp. in bat bugs \textit{Cimex pipistrelli} (Hemiptera: Cimicidae), Central Europe. \textit{Parasitology Research}. 2022; 121(11): 3341--3345. DOI: \href{https://doi.org/10.1007/s00436-022-07668-4}{10.1007/s00436-022-07668-4}

\item Goodrich I, \textbf{McKee C}, Margos G, Kosoy M. Molecular characterization of a novel relapsing fever \textit{Borrelia} species from the desert cottontail (\textit{Sylvilagus audubonii}) in New Mexico, USA. \textit{Journal of Wildlife Diseases}. 2022; 58(3): 646--651. DOI: \href{https://doi.org/10.7589/JWD-D-21-00148}{10.7589/JWD-D-21-00148}

\item \textbf{McKee CD}\textsuperscript{*\dag}, Islam A\textsuperscript{\dag}, Rahman MZ, Khan SU, Rahman M, Satter SM, Islam A, Yinda CK, Epstein JH, Daszak P, Munster VJ. Nipah virus detection at bat roosts after spillover events, Bangladesh, 2012–2019. \textit{Emerging Infectious Diseases}. 2022; 28(7): 1384--1392. DOI: \href{https://doi.org/10.3201/eid2807.212614}{10.3201/eid2807.212614}

\item Rice BL\textsuperscript{\dag}, Lessler J\textsuperscript{\dag}, \textbf{McKee C}\textsuperscript{\dag}, Metcalf CJE\textsuperscript{\dag}. Why do some coronaviruses become pandemic threats when others do not? \textit{PLOS Biology}. 2022; 20(5): e3001652. DOI: \href{https://doi.org/10.1371/journal.pbio.3001652}{10.1371/journal.pbio.3001652}

\item Ruiz-Aravena M\textsuperscript{\dag}, \textbf{McKee C}\textsuperscript{\dag}, Gamble A, Lunn T, Morris A, Snedden CE, Yinda CK, Port JR, Buchholz DW, Yeo YY, Faust C, Jax E, Dee L, Jones DN, Kessler MK, Falvo C, Crowley D, Bharti N, Brook CE, Aguilar HC, Peel AJ, Restif O, Schountz T, Parrish CR, Gurley ES, Lloyd-Smith JO, Hudson PJ, Munster VJ, Plowright RK. Ecology, evolution and spillover of coronaviruses from bats. \textit{Nature Reviews Microbiology}. 2022; 20: 299--314. DOI: \href{https://doi.org/10.1038/s41579-021-00652-2}{10.1038/s41579-021-00652-2}

\item Redd AD, Peetluk LS, Jarrett BA, Hanrahan C, Schwartz S, Rao A, Jaffe AE, Peer AD, Jones CB, Lutz CS, \textbf{McKee CD}, Patel EU, Rosen JG, Garrison Desany H, McKay HS, Muschelli J, Andersen KM, Link MA, Wada N, Baral P, Young R, Boon D, Grabowski MK, Gurley ES, the Novel Coronavirus Research Compendium Team. Curating the evidence about COVID-19 for frontline public health and clinical care: the Novel Coronavirus Research Compendium. \textit{Public Health Reports}. 2022; 137(2): 197--202. DOI: \href{https://doi.org/10.1177/00333549211058732}{10.1177/00333549211058732}

\item Islam A, \textbf{McKee C}, Ghosh PK, Abedin J, Epstein JH, Daszak P, Luby SP, Khan SU, Gurley ES. Seasonality of date palm sap feeding behavior by bats in Bangladesh. \textit{EcoHealth}. 2021; 18: 359--371. DOI: \href{https://doi.org/10.1007/s10393-021-01561-9}{10.1007/s10393-021-01561-9}

\item Zorrilla VO, Lozano ME, Espada LJ, Kosoy M, \textbf{McKee C}, Valdivia HO, Arevalo H, Troyes M, Stoops CA, Fisher ML, Vásquez GM. Comparison of sand fly trapping approaches for vector surveillance of \textit{Leishmania} and \textit{Bartonella} species in ecologically distinct, endemic regions of Peru. \textit{PLOS Neglected Tropical Diseases}. 2021; 15(7): e0009517. DOI: \href{https://doi.org/10.1371/journal.pntd.0009517}{10.1371/journal.pntd.0009517}

\item \textbf{McKee CD}\textsuperscript{*}, Islam A, Luby SP, Salje H, Hudson PJ, Plowright RK, Gurley ES. The ecology of Nipah virus in Bangladesh: a nexus of land-use change and opportunistic feeding behavior in bats. \textit{Viruses}. 2021; 13(2): 169. DOI: \href{https://doi.org/10.3390/v13020169}{10.3390/v13020169}

\item \textbf{McKee C}\textsuperscript{*}, Bai Y, Webb C, Kosoy M. Bats are key hosts in the radiation of mammal-associated \textit{Bartonella} bacteria. \textit{Infection, Genetics and Evolution}. 2021; 89: 104719. DOI: \href{https://doi.org/10.1016/j.meegid.2021.104719}{10.1016/j.meegid.2021.104719}

\item Goodrich I, \textbf{McKee C}, Kosoy M. \textit{Trypanosoma} (\textit{Herpetosoma}) diversity in rodents and lagomorphs of New Mexico with a focus on epizootological aspects of infection in Southern Plains woodrats (\textit{Neotoma micropus}). \textit{PLOS ONE}. 2020; 15(12): e0244803. DOI: \href{https://doi.org/10.1371/journal.pone.0244803}{10.1371/journal.pone.0244803}

\item Rudolf I, Blažejová H, Mendel J, Straková P, Šebesta O, Rettich F, Čabanová V, Miterpáková M, Betášová L, Peško J, Barbušinová E, \textbf{McKee C}, Osikowicz L, Šikutová S, Hubálek Z, Kosoy M. \textit{Bartonella} species in medically important mosquitoes, Central Europe. \textit{Parasitology Research}. 2020; 119(8): 2713--2717. DOI: \href{https://doi.org/10.1007/s00436-020-06732-1}{10.1007/s00436-020-06732-1}

\item Goodrich I, \textbf{McKee C}, Kosoy M. Longitudinal study of bacterial infectious agents in a community of small mammals in New Mexico. \textit{Vector-Borne and Zoonotic Diseases}. 2020; 20(7): 496--508. DOI: \href{https://doi.org/10.1089/vbz.2019.2550}{10.1089/vbz.2019.2550}

\item \textbf{McKee CD}\textsuperscript{*}, Krawczyk AI, Sándor AD, Görföl T, Földvári M, Földvári G, Dekeukeleire D, Haarsma A-J, Kosoy MY, Webb CT, Sprong H. Host phylogeny, geographic overlap, and roost sharing shape parasite communities in European bats. \textit{Frontiers in Ecology and Evolution}. 2019; 7: 69. DOI: \href{https://doi.org/10.3389/fevo.2019.00069}{10.3389/fevo.2019.00069}

\item Bai Y, Osinubi MOV, Osikowicz L, \textbf{McKee C}, Vora NM, Rizzo MR, Recuenco S, Davis L, Niezgoda M, Ehimiyein AM, Kia GSN, Oyemakinde A, Adeniyi OS, Gbadegesin YH, Saliman OA, Ogunniyi A, Ogunkoya AB, Kosoy MY, Idanre Bat Festival Investigation Team. Human exposure to novel \textit{Bartonella} species from contact with fruit bats. \textit{Emerging Infectious Diseases}. 2018; 24(12): 2317--2323. DOI: \href{https://doi.org/10.3201/eid2412.181204}{10.3201/eid2412.181204}

\item Kellner A, Carver S, Scorza V, \textbf{McKee CD}, Lappin M, Crooks KR, VandeWoude S, Antolin MF. Transmission pathways and spillover of an erythrocytic bacterial pathogen from domestic cats to wild felids. \textit{Ecology and Evolution}. 2018; 8(19): 9779--9792. DOI: \href{https://doi.org/10.1002/ece3.4451}{10.1002/ece3.4451}

\item \textbf{McKee CD}\textsuperscript{*}, Osikowicz LM, Schwedhelm TR, Maes SE, Enscore RE, Gage KL, Kosoy MY. Acquisition of \textit{Bartonella elizabethae} by experimentally exposed oriental rat fleas (\textit{Xenopsylla cheopis}; Siphonaptera, Pulicidae) and excretion of \textit{Bartonella} DNA in flea feces. \textit{Journal of Medical Entomology} 2018; 55(5): 1292--1298. DOI: \href{https://doi.org/10.1093/jme/tjy085}{10.1093/jme/tjy085}

\item Kosoy M\textsuperscript{\dag}, \textbf{McKee C}\textsuperscript{\dag}, Albayrak L, Fofanov Y. Genotyping of \textit{Bartonella} bacteria and their animal hosts: current status and perspectives. \textit{Parasitology}. 2018; 145(5): 543--562. DOI: \href{https://doi.org/10.1017/S0031182017001263}{10.1017/S0031182017001263}

\item Gorsich EE, \textbf{McKee CD}, Grear DA, Miller RS, Portacci K, Lindström T, Webb CT. Model-guided suggestions for targeted surveillance based on cattle shipments in the U.S. \textit{Preventive Veterinary Medicine}. 2018; 150: 52--59. DOI: \href{https://doi.org/10.1016/j.prevetmed.2017.12.004}{10.1016/j.prevetmed.2017.12.004}

\item \textbf{McKee CD}\textsuperscript{*}, Osikowicz LM, Schwedhelm TR, Bai Y, Castle KT, Kosoy MY. Survey of parasitic bacteria in bat bugs, Colorado. \textit{Journal of Medical Entomology.} 2018; 55(1): 237--241. DOI: \href{https://doi.org/10.1093/jme/tjx155}{10.1093/jme/tjx155}

\item \textbf{McKee CD}\textsuperscript{*}, Kosoy MY, Bai Y, Osikowicz LM, Franka R, Gilbert AT, Boonmar S, Rupprecht CE, Peruski LF. Diversity and phylogenetic relationships among \textit{Bartonella} strains from Thai bats. \textit{PLOS ONE}. 2017; 12(7): e0181696. DOI: \href{https://doi.org/10.1371/journal.pone.0181696}{10.1371/journal.pone.0181696}

\item Urushadze L, Bai Y, Osikowicz L, \textbf{McKee C}, Sidamonidze K, Putkaradze D, Imnadze P, Kandaurov A, Kuzmin I, Kosoy M. Prevalence, diversity, and host associations of \textit{Bartonella} strains in bats from Georgia (Caucasus). \textit{PLOS Neglected Tropical Diseases}. 2017; 11(4): e0005428. DOI: \href{https://doi.org/10.1371/journal.pntd.0005428}{10.1371/journal.pntd.0005428}

\item Bai Y, Urushadze L, Osikowicz L, \textbf{McKee C}, Kuzmin I, Kandaurov A, Babuadze G, Natradze I, Imnadze P, Kosoy M. Molecular survey of bacterial zoonotic agents in bats from the country of Georgia (Caucasus). \textit{PLOS ONE}. 2017; 12(1): e0171175. DOI: \href{https://doi.org/10.1371/journal.pone.0171175}{10.1371/journal.pone.0171175}

\item \textbf{McKee CD}\textsuperscript{*}, Hayman DTS, Kosoy MY, Webb CT. Phylogenetic and geographic patterns of bartonella host shifts among bat species. \textit{Infection, Genetics and Evolution}. 2016; 44: 382--394. DOI: \href{https://doi.org/10.1016/j.meegid.2016.07.033}{10.1016/j.meegid.2016.07.033}

\item Zinzow-Kramer WM, Horton BM, \textbf{McKee CD}, Michaud JM, Tharp GK, Thomas JW, Tuttle EM, Yi S, Maney DL. Genes located in a chromosomal inversion are correlated with territorial song in white-throated sparrows. \textit{Genes, Brain and Behavior}. 2015; 14(8): 641--654. DOI: \href{https://doi.org/10.1111/gbb.12252}{10.1111/gbb.12252}

\item Bai Y, Hayman DTS, \textbf{McKee CD}, Kosoy MY. Classification of \textit{Bartonella} strains associated with straw-colored fruit bats (\textit{Eidolon helvum}) across Africa using a multi-locus sequence typing platform. \textit{PLOS Neglected Tropical Diseases}. 2015; 9(1): e0003478. DOI: \href{https://doi.org/10.1371/journal.pntd.0003478}{10.1371/journal.pntd.0003478}

\end{pubenum}

\subsection*{\textit{Preprint Articles}}

\begin{pubenum}

\item Cortés Azuero O, Lefrancq N, Nikolay B, McKee C, Cappelle J, Hul V, Ou TP, Hoem T, Lemey P, Rahman MZ, Islam A., Gurley ES, Duong V, Salje H. The genetic diversity of Nipah virus across spatial scales. \textit{medRxiv}. 2023. DOI: \href{https://doi.org/10.1101/2023.07.14.23292668}{2023.07.14.23292668}

\item Howerton E, Contamin L, Mullany LC, Qin MM, Reich NG, Bents SJ, Borchering RK, Jung SM, Loo SL, Smith CP, Levander J, ..., \textbf{McKee C}, ..., Viboud C, Lessler J. Informing pandemic response in the face of uncertainty. An evaluation of the US COVID-19 Scenario Modeling Hub. \textit{medRxiv}. 2023. DOI: \href{https://doi.org/10.1101/2023.06.28.23291998}{10.1101/2023.06.28.23291998v1}

\item \textbf{McKee CD}\textsuperscript{*}, Webb CT, Kosoy MY, Bai Y, Osikowicz LM, Suu-Ire R, Ntiamoa-Baidu Y, Cunningham AA, Wood JL, Hayman DT. Manipulating vector transmission reveals local processes in bacterial communities of bats. \textit{bioRxiv}. 2021. DOI: \href{https://doi.org/10.1101/2021.03.03.433743}{10.1101/2021.03.03.433743}

\end{pubenum}

%----------------------------------------------------------------------------------------
%	PRACTICE ACTIVITIES
%----------------------------------------------------------------------------------------

\section*{PRACTICE ACTIVITIES}

% \subsection*{Practice-Related Reports}

\subsection*{Presentations to Policymakers, Communities, and Other Stakeholders}

\href{https://usamm-gen-net.shinyapps.io/usamm-gen-net/}{USAMM R Shiny Visualization} $\cdot$ Developed tool for visualizing estimated county-to-county animal shipments in the US $\cdot$ Presented to stakeholders at USDA APHIS (2015)

\subsection*{Media Dissemination}

\href{https://ncrc.jhsph.edu/}{2019 Novel Coronavirus Research Compendium (NCRC)} $\cdot$ Curated and wrote public-facing summaries and assessments of research articles on the Ecology \& Spillover of SARS-CoV-2 and related coronaviruses (2020--2021)

\subsection*{Media Coverage}

\href{https://youtu.be/p6I0kRVGyo4}{Disease on the Wing} $\cdot$ Youtube video produced by \textit{The Scientist} magazine (2014)

\href{https://www.the-scientist.com/features/lurking-in-the-shadows-36302}{Lurking in the Shadows} $\cdot$ Feature article in \textit{The Scientist} magazine (2014)

%----------------------------------------------------------------------------------------
%	SOFTWARE
%----------------------------------------------------------------------------------------

\section*{SOFTWARE AND TEMPLATES}

% \subsection*{GitHub Templates}

% DaSL Collection - \textit{An automatically generating resource documenting all completed Data Science Lab content.} [\href{https://github.com/fhdsl/DaSL_Collection}{Available on GitHub}]

% Fred Hutch Letterhead - \textit{A LaTeX template for Fred Hutch-themed letterhead.} [\href{https://github.com/fhdsl/FH_letterhead}{Available on GitHub}]

% Online Tools for Training Resources (OTTR) Template - \textit{A GitHub template that simplifies and accelerates publishing course content in bookdown format or to Leanpub and Coursera. Created with Candace Savonen, Carrie Wright, and others.} [\href{https://github.com/jhudsl/OTTR_Template}{Available on GitHub}]

% AnVIL Template - \textit{A GitHub template variation of the OTTR Template that automatically formats and generates content specific to the AnVIL Project. Created with Katherine Cox.} [\href{https://github.com/jhudsl/AnVIL_Template}{Available on GitHub}]

\subsection*{Software \& Tools}

The Flexible Epidemic Modeling Pipeline (\href{https://www.flepimop.org/}{flepiMoP}) $\cdot$ A software suite for simulating a wide range of compartmental models of infectious disease transmission [\href{https://github.com/HopkinsIDD/flepiMoP}{Available on GitHub}]

%----------------------------------------------------------------------------------------
% Break page and add Part II header

\newpage
\name{CLIFTON D. MCKEE}
\parttwo

%----------------------------------------------------------------------------------------

%----------------------------------------------------------------------------------------
%	TEACHING
%----------------------------------------------------------------------------------------

\section*{TEACHING}

% \subsection*{Capstone Advisees}

\subsection*{Research Advisees}

% \subsubsection*{Postdoctoral Fellows}

\subsubsection*{Master's Students}

Niu, Yannan $\cdot$ Master of Public Health Student Research Assistant $\cdot$ Johns Hopkins Bloomberg School of Public Health (2023 -- present)

Garcia, Andrés $\cdot$ Master of Health Science Student Research Assistant $\cdot$ Johns Hopkins Bloomberg School of Public Health (2023 -- present)

Jackson, Jules $\cdot$ Master of Science Student Research Assistant $\cdot$ Johns Hopkins Bloomberg School of Public Health (2022--2023)

Endres, Kelly $\cdot$ Master of Science in Public Health Student Research Assistant $\cdot$ Johns Hopkins Bloomberg School of Public Health (2021--2022)

\subsubsection*{Undergraduate Students}

Lazarus, Paul $\cdot$ Research Assistant $\cdot$ Colorado State University (2018) \\
Cannella, Alexis $\cdot$ Research Assistant $\cdot$ Colorado State University (2016--2017) \\
Doyle, Jake $\cdot$ Research Assistant $\cdot$ Colorado State University (2014--2015) \\
Leming, Madison $\cdot$ Research Assistant $\cdot$ Colorado State University (2014--2015)

\subsection*{Classroom Instruction -- Instructor of Record}

\subsubsection*{Johns Hopkins University}

\href{https://jhudatascience.org/intro_to_r/}{Introduction to \texttt{R} for Public Health Researchers} (140.604.79) $\cdot$ Summer Institute 2023 (enrollment: 35)

The One Health Approach to Epidemiology and Global Public Health (340.610.81) $\cdot$ Spring 4th Term 2023 (enrollment: 35)

\subsection*{Classroom Instruction -- Teaching Assistant}

\subsubsection*{Johns Hopkins Bloomberg School of Public Health}

Topics in Infectious Disease Epidemiology (340.668.89) $\cdot$ Teaching Assistant (2022)

The One Health Approach to Epidemiology and Global Public Health (340.610.81) $\cdot$ Teaching Assistant (2021 \& 2022)

Epidemiology of Infectious Diseases (340.627.01) $\cdot$ Teaching Assistant (2020 \& 2021)

\subsubsection*{Colorado State University}

Molecular and General Genetics (BZ 350) $\cdot$ Teaching Assistant \& Recitation Instructor (2014, 2018 \& 2019)

Introduction to Evolution (BZ 220) $\cdot$ Teaching Assistant (2018)

Ecology (LIFE 320) $\cdot$ Teaching Assistant (2017)

Ornithology (BZ 335) $\cdot$ Teaching Assistant \& Laboratory Instructor (2014 \& 2015)

Attributes of Living Systems (LIFE 102) $\cdot$ Teaching Assistant \& Laboratory Instructor (2013)

\subsection*{Short Courses \& Workshops}

Coordinator \& Content Developer $\cdot$ Applied Modeling in Public Health workshop $\cdot$ Johns Hopkins Bloomberg School of Public Health (2022 -- present)

Content Developer \& Co-Instructor $\cdot$ Zombiecology: Workshop on Disease Ecology $\cdot$ National Science Olympiad $\cdot$ Colorado State University (2018)

% \href{http://sisbid.github.io/Data-Wrangling/}{Data Wrangling with R}, University of Washington Summer Institute (2022, 2021)

\subsection*{Other Teaching}

Evolution \& medicine: how evolution influences diseases \& our response to them $\cdot$ Guest lecture in Introduction to Evolution (BZ 220) $\cdot$ Colorado State University (2018)

Phylogenetic analysis of pathogens $\cdot$ Guest lecture in Principles of Systematic Zoology (BZ 424) $\cdot$ Colorado State University (2018)

Vampires, rabies, and bugs -- Oh my! $\cdot$ Guest lecture in Tropical Ecology and Evolution $\cdot$ Colorado State University (2017)

Bats and their parasites $\cdot$ Guest lecture in Ecology of Infectious Diseases (BZ 418) $\cdot$ Colorado State University (2016)

Prokaryotic genetics $\cdot$ Guest lecture in Molecular and General Genetics (BZ 350) $\cdot$ Colorado State University (2014)

% Data Visualization using R and ggplot (guest lecture, Colorado State University, 2016)

% \subsection*{Educational Resources}

% \subsubsection*{Analysis Visualization and Informatics Lab-space (AnVIL) Project}

% Created course materials for users to understand and better leverage the \href{https://anvilproject.org/}{AnVIL cloud computing platform} for education and research, including: 

% \begin{itemize}

% \item \href{https://hutchdatascience.org/AnVIL_Book_Epigenetics_Intro/}{AnVIL Epigenetics Introduction}: An introduction to analysis of epigenetic data and epigenetics concepts on AnVIL. Created with Ifeoma Nwigwe.

% \item \href{https://www.youtube.com/watch?v=tVh93e6TzCE\&list=PL6aYJ_0zJ4uCABkMngSYjPo_3c-nUUmio}{AnVIL Shorts: Learn about AnVIL in 2 Minutes}: a series of two-minute videos for new users to quickly understand multiple concepts and personas on AnVIL.

% \item \href{https://jhudatascience.org/AnVIL_Book_Getting_Started/}{Getting Started on AnVIL}: a series of step-by-step guides for setting up accounts focused on three personas: PIs, Analysts, and Consortia. Also includes custom videos created using JHU software packages.

% \item \href{https://jhudatascience.org/AnVIL_Book_Instructor_Guide/}{AnVIL Instructor Guide}: a guide to help classroom instructors who are new to AnVIL set up their accounts and start developing content.

% \end{itemize}

% \subsubsection*{Genomic Data Science Community Network (GDSCN) Project}

% Created course materials for \href{https://www.gdscn.org/}{GDSCN} faculty to use in their classrooms, including: 

% \begin{itemize}

% \item \href{https://jhudatascience.org/GDSCN_Book_Statistics_for_Genomics_Differential_Expression/}{Statistics for Genomics: Differential Expression}: A set of lab modules in the R programming language for an introduction to differential gene expression.

% \item \href{https://jhudatascience.org/GDSCN_Book_SARS_Galaxy_on_AnVIL/}{GDSCN Book: SARS with Galaxy on AnVIL}: a series of resources for instructors to engage students in a cloud-based Galaxy activity on AnVIL, focused on SARS-CoV-2 variant detection. Includes instructional videos, background lectures for instructors to use in their own classes, and a student lab activity guide.

% \item \href{https://anvil.terra.bio/#workspaces/gdscn-exercises/SARS-CoV-2-Genome}{AnVIL Workspace} for GDSCN members to use to follow the SARS-CoV-2 variant activity.

% \end{itemize}

%----------------------------------------------------------------------------------------
%	RESEARCH GRANT PARTICIPATION
%----------------------------------------------------------------------------------------

\section*{RESEARCH GRANT PARTICIPATION}

\subsection*{Current Support}

Project Title: Safety and Healthcare Epidemiology Prevention Research Development (SHEPheRD) (75D30121F00005) \\
Dates: 09/03/2021 -- 09/02/2024 \\
Sponsoring Agency: Centers for Disease Control \\
Principal Investigator: Shaun Truelove \\
Main Grant Objective: Lead recruitment and logistics for Applied Modeling in Public Health workshop series; assist with workshop content and Coursera course development \\
Role: Research Associate

Project Title: Solving Opportunities for Spillover (SOS): Frequency and Mechanisms of Cross-species Transmission of Henipaviruses in Bangladesh (1U01AI168287-01A1) \\
Dates: 01/03/2023 -- 01/02/2028 \\
Sponsoring Agency: NIH/NIAID \\
Principal Investigator: Emily Gurley \\
Main Grant Objective: Provide support for field work, data analysis, and writing manuscripts; draft protocols for approval, collate and analyze data coming from the field and the laboratory, and lead development of manuscripts for publication \\
Role: Research Associate

\subsection*{Past Support}

Project Title: Preventing emergence and spillover of bat pathogens in high risk global hotspots (G166-19-W7329) \\
Dates: 10/01/2018 -- 09/30/2022 \\
Sponsoring Agency: DARPA/Montana State University \\
Principal Investigator: Raina Plowright (Project Director), Emily Gurley (Site PI) \\
Main Grant Objective: Work with the team in Bangladesh to develop and implement protocols to sample bats, including identifying roosts best suited for sampling, ensuring protocols are aligned with broad aims of this grant, and that work is delivered on time; real-time analysis of virus detection data \\
Role: Postdoctoral Fellow

Project Title: COVID-19 Technical Lead in the Implementation of the Strategic Preparedness (202548990) \\
Dates: 08/03/2022 -- 11/03/2022 \\
Sponsoring Agency: World Health Organization \\
Principal Investigator: Emily Gurley \\
Main Grant Objective: Support COVID-19 technical lead to improve evidence-based guidance by completing a systematic review of COVID-19 superspreading \\
Role: Postdoctoral Fellow

%----------------------------------------------------------------------------------------
%	ACADEMIC SERVICE
%----------------------------------------------------------------------------------------

% \section*{ACADEMIC SERVICE}

% Executive Committee student member, Graduate Degree Program in Ecology, Colorado State University (7/1/15-6/30/16)

% Faculty search committee student organizer, Department of Biology, Colorado State University (2015)

%----------------------------------------------------------------------------------------
%	PRESENTATIONS
%----------------------------------------------------------------------------------------

\section*{PRESENTATIONS}

\subsection*{Scientific Meetings}

\subsubsection*{Oral Presentations}

Comprehensive time tree analysis identifies bats as key to the radiation of mammal-associated \textit{Bartonella} bacteria $\cdot$ Evolution $\cdot$ Providence, RI (2019)

Timing the diversification of a mammal parasite, \textit{Bartonella} $\cdot$ Ecological Society of America $\cdot$ New Orleans, LA (2018)

Timing the diversification of a mammal parasite, \textit{Bartonella} $\cdot$ Front Range Student Ecology Symposium $\cdot$ Fort Collins, CO (2018)

Long-term monitoring of \textit{Bartonella} spp. bacteria in a captive colony of fruit bats and experimental evidence of bat flies as vectors of bartonella $\cdot$ Ecological Society of America $\cdot$ Portland, OR (2017)

Long-term monitoring of \textit{Bartonella} spp. bacteria in a captive colony of fruit bats and experimental evidence of bat flies as vectors of bartonella $\cdot$ International Symposium on Infectious Diseases of Bats $\cdot$ Fort Collins, CO (2017)

Host phylogenetic distance and ectoparasite overlap predict \textit{Bartonella} sharing in European bats $\cdot$ Front Range Student Ecology Symposium $\cdot$ Fort Collins, CO (2017)

Phylogeography of \textit{Bartonella} bacteria in \textit{Eidolon} spp. fruit bats across Africa $\cdot$ Ecological Society of America $\cdot$ Baltimore, MD (2015)

Phylogeography of \textit{Bartonella} in \textit{Eidolon} fruit bats across Africa $\cdot$ Front Range Student Ecology Symposium $\cdot$ Fort Collins, CO (2015)

\subsubsection*{Poster Presentations}

Nipah virus detection at bat roosts following spillover events in Bangladesh, 2012–2019 $\cdot$ International Conference on Emerging Infectious Diseases $\cdot$ Atlanta, GA (2022)

Manipulating vector transmission reveals local processes in bacterial communities of bats $\cdot$ Ecology and Evolution of Infectious Disease $\cdot$ Princeton, NJ (2019)

Dispersal of hosts and vectors predicts bacterial community structure across an island chain $\cdot$ Front Range Student Ecology Symposium $\cdot$ Fort Collins, CO (2019)

Timing the diversification of a mammal parasite, \textit{Bartonella} $\cdot$ Colorado State University Graduate Student Showcase $\cdot$ Fort Collins, CO (2017)

Linking patterns of bacterial parasite diversity across host and vector communities $\cdot$ Ecology and Evolution of Infectious Disease $\cdot$ Santa Barbara, CA (2017)

Phylogeography of \textit{Bartonella} bacteria in \textit{Eidolon} spp. fruit bats across Africa $\cdot$ Ecology and Evolution of Infectious Disease $\cdot$ Athens, GA (2015)

Large-scale patterns of \textit{Bartonella} prevalence and diversity in African fruit bats $\cdot$ Ecological Society of America $\cdot$ Sacramento, CA (2014)

Large-scale patterns of \textit{Bartonella} prevalence and diversity in African fruit bats $\cdot$ Ecology and Evolution of Infectious Disease $\cdot$ Fort Collins, CO (2014)

A novel method of isolating multi-locus sequence data for characterizing \textit{Bartonella} diversity in bats $\cdot$ Front Range Student Ecology Symposium $\cdot$ Fort Collins, CO (2014)

\subsection*{Invited Seminars}

Investigating Nipah virus spillover at the human-animal interface in Bangladesh $\cdot$ Cornell University College of Veterinary Medicine $\cdot$ Ithaca, NY (2023)

Global change, infectious disease, and public health: the need for ecological interventions $\cdot$ WHO GOARN/RCCE Collective Service $\cdot$ Virtual (2022)

Bad roommates? Inferring bacterial interactions in coinfected individuals $\cdot$ Colorado State University Vice President Office of Research Symposium $\cdot$ Fort Collins, CO (2018)

%----------------------------------------------------------------------------------------
%	ADDITIONAL INFORMATION 
%----------------------------------------------------------------------------------------

\section*{ADDITIONAL INFORMATION}

\subsection*{Personal Statement}

I am an infectious disease ecologist interested in zoonotic pathogens and their dynamics within host populations and at the human-animal interface. Overall, my research sits firmly within One Health, seeking to understand disease emergence at the intersection between public health, animal ecology, and environmental change. I use a combination of fieldwork, statistical modeling, molecular genetics, and phylogenetics to investigate pathogen persistence in animal reservoirs, evolution of host specificity, and ecological drivers of pathogen spillover.

\subsection*{Keywords}

disease ecology, epidemiology, data science, zoonoses, One Health, phylogenetics, evolution

\end{document}