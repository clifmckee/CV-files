% Written by Ava Hoffman
% Please use as you like, but it would be nice if you credited me :)
% 
% Use LaTeX to compile and ensure style file (cv.cls) is in the same directory.
%
%%%%%%%%%%%%%%%%%%%%%%%%%%%%%%%%%%%%%%%%%

%----------------------------------------------------------------------------------------
%	PREAMBLE
%----------------------------------------------------------------------------------------

\documentclass{cv}

\begin{document}

%----------------------------------------------------------------------------------------
%	PROFESSIONAL DATA
%----------------------------------------------------------------------------------------

\name{CLIFTON D. MCKEE}

\section*{PROFESSIONAL DATA}

\subsection*{Contact Information}

Johns Hopkins University \\
Bloomberg School of Public Health \\
Department of Epidemiology \\
615 N. Wolfe St., Room E6008\\
Baltimore, MD 21205 \\
Email: \emailcontact{cmckee7@jhu.edu} $\cdot$ \emailcontact{clifton.mckee@gmail.com} \\
Phone: 970-889-9540

\subsection*{Website \& Social Media}

Website: \href{https://clifmckee.github.io/}{clifmckee.github.io} \\
GitHub: \href{https://github.com/clifmckee/}{clifmckee} \\
Twitter/X: \href{https://twitter.com/clifmckee/}{clifmckee}

%----------------------------------------------------------------------------------------
%	EDUCATION AND TRAINING
%----------------------------------------------------------------------------------------

\section*{EDUCATION AND TRAINING}

\subsection*{Degrees}

Ph.D. / 2020 $\cdot$ Ecology $\cdot$ Colorado State University $\cdot$ Fort Collins, CO \\
Thesis: Evolutionary and ecological processes in microparasite communities of bats \\
Supervisor: Prof. Colleen Webb

M.S. / 2015 $\cdot$ Ecology $\cdot$ Colorado State University $\cdot$ Fort Collins, CO \\
Thesis: Spatial, demographic, and phylogenetic patterns of \textit{Bartonella} diversity in bats \\
Supervisor: Prof. Colleen Webb

B.S. / 2011 $\cdot$ Ecology and Evolution $\cdot$ University of Pittsburgh $\cdot$ Pittsburgh, PA \\
B.A. / 2011 $\cdot$ Environmental Studies $\cdot$ University of Pittsburgh $\cdot$ Pittsburgh, PA \\
\textit{magna cum laude}

\subsection*{Postdoctoral Training}

2020--2022 $\cdot$ Johns Hopkins Bloomberg School of Public Health $\cdot$ Baltimore, MD \\
Member: \href{http://www.iddynamics.jhsph.edu/}{Infectious Disease Dynamics} $\cdot$ \href{https://batonehealth.org/}{Bat One Health} \\
Supervisor: Prof. Emily Gurley

%----------------------------------------------------------------------------------------
%	PROFESSIONAL EXPERIENCE
%----------------------------------------------------------------------------------------

\section*{PROFESSIONAL EXPERIENCE}

\subsection*{Johns Hopkins Bloomberg School of Public Health}

Faculty Research Associate $\cdot$ Department of Epidemiology (2023 -- present)

Postdoctoral Fellow $\cdot$ Department of Epidemiology (2020--2022)

\subsection*{CDC Division of Vector-Borne Diseases}

Research Fellow $\cdot$ Bacterial Diseases Branch (2015--2017)

%----------------------------------------------------------------------------------------
%	PUBLICATIONS
%----------------------------------------------------------------------------------------

\section*{PUBLICATIONS}

\subsection*{Journal Articles (Peer Reviewed)}

{\small \textsuperscript{*}Corresponding author, \textsuperscript{\dag}Equal contribution, \textsuperscript{\ddag}Student advisee}

\begin{pubenum}

\item \textbf{McKee CD}, Yu EX, Garcia A, Jackson J, Koyuncu A, Rose S, Azman AS, Lobner K, Sacks E, Van Kerkhove MD, Gurley ES. Superspreading of SARS-CoV-2: a systematic review and meta-analysis of event attack rates and individual transmission patterns. \textit{In press at Epidemiology \& Infection}. DOI: \href{https://doi.org/10.1101/2024.01.25.24301669}{10.1101/2024.01.25.24301669}

\item \textbf{McKee CD}\textsuperscript{*}, Peel AJ, Hayman DTS, Suu-Ire R, Ntiamoa-Baidu Y, Cunningham AA, Wood JLN, Webb CT, Kosoy MY. Ectoparasite and bacterial population genetics and community structure indicate extent of bat movement across an island chain. \textit{In press at Parasitology}. DOI: \href{https://doi.org/10.1017/S0031182024000660}{10.1017/S0031182024000660}.

\item Mathis SM, Webber AE, León T, Murray EL, Sun M, White LA, Brooks LC, Green A, Hu AJ, Rosenfeld R, Shemetov D, …, \textbf{McKee CD}, …, Borchering RK. Evaluation of FluSight influenza forecasting in the 2021–22 and 2022–23 seasons with a new target laboratory-confirmed influenza hospitalizations. \textit{Nature Communications}. 2024;15:6289. DOI: \href{https://doi.org/10.1038/s41467-024-50601-9}{10.1038/s41467-024-50601-9}

\item Cortes-Azuero O, Lefrancq N, Nikolay B, \textbf{McKee C}, Cappelle J, Hul V, Ou TP, Hoem T, Lemey P, Rahman MZ, Islam A, Gurley ES, Duong V, Salje H. The genetic diversity of Nipah virus across spatial scales. \textit{The Journal of Infectious Diseases}. 2024; jiae221. DOI: \href{https://doi.org/10.1093/infdis/jiae221}{10.1093/infdis/jiae221}

\item Jung S-m, Loo SL, Howerton E, Contamin L, Smith CP, Carcelén EC, Yan K, Bents SJ, Levander J, Espino J, Lemaitre JC, …, \textbf{McKee CD}, …, Viboud C, Lessler J. Potential impact of annual vaccination with reformulated COVID-19 vaccines: lessons from the US COVID-19 scenario modeling hub. \textit{PLOS Medicine}. 2024; 21(4): e1004387. DOI: \href{https://doi.org/10.1371/journal.pmed.1004387}{10.1371/journal.pmed.1004387}

\item Lemaitre JC, Loo SL, Kaminsky J, Lee EC, \textbf{McKee C}, Smith C, Jung S-m, Sato K, Carcelen E, Hill A, Lessler J, Truelove S. \textit{flepiMoP}: the evolution of a flexible infectious disease modeling pipeline during the COVID-19 pandemic. \textit{Epidemics}. 2024; 47: 100753. DOI: \href{https://doi.org/10.1016/j.epidem.2024.100753}{10.1016/j.epidem.-2024.100753}

\item Howerton E, Contamin L, Mullany LC, Qin MM, Reich NG, Bents SJ, Borchering RK, Jung SM, Loo SL, Smith CP, Levander J, …, \textbf{McKee C,} …, Viboud C, Lessler J. Evaluation of the US COVID-19 Scenario Modeling Hub for informing pandemic response under uncertainty. \textit{Nature Communications}. 2023; 14: 7260. DOI: \href{https://doi.org/10.1038/s41467-023-42680-x}{10.1038/s41467-023-42680-x}

\item Fagre AC, Islam A, Reeves WK, Kading RC, Plowright RK, Gurley ES, \textbf{McKee CD}\textsuperscript{*}. \textit{Bartonella} infection in fruit bats and bat flies, Bangladesh. \textit{Microbial Ecology}. 2023; 86: 2910–2922. DOI: \href{https://doi.org/10.1007/s00248-023-02293-9}{10.1007/s00248-023-02293-9}

\item Szentiványi T, \textbf{McKee C}, Jones G, Foster JT. Trends in bacterial pathogens of bats: global distribution and knowledge gaps. \textit{Transboundary and Emerging Diseases}. 2023; 9285855. DOI: \href{https://doi.org/10.1155/2023/9285855}{10.1155/2023/9285855}

\item Seidlova V, Straková P, Kejíková R, Nemcova M, Bartonička T, Salát J, Dufková L, Šikutová S, Mendel J, \textbf{McKee C}, Zukal J, Pikula J, Rudolf I. Detection of \textit{Leptospira} species in bat cadavers, Czech and Slovak Republics. \textit{Emerging Microbes \& Infections}. 2022; 11(1): 2211--2213. DOI: \href{https://doi.org/10.1080/22221751.2022.2117095}{10.1080/22221751.2022.2117095}

\item Kejíková R, \textbf{McKee C}, Straková P, Šikutová S, Mendel J, Rudolf I. First detection of \textit{Bartonella} spp. in bat bugs \textit{Cimex pipistrelli} (Hemiptera: Cimicidae), Central Europe. \textit{Parasitology Research}. 2022; 121(11): 3341--3345. DOI: \href{https://doi.org/10.1007/s00436-022-07668-4}{10.1007/s00436-022-07668-4}

\item Goodrich I, \textbf{McKee C}, Margos G, Kosoy M. Molecular characterization of a novel relapsing fever \textit{Borrelia} species from the desert cottontail (\textit{Sylvilagus audubonii}) in New Mexico, USA. \textit{Journal of Wildlife Diseases}. 2022; 58(3): 646--651. DOI: \href{https://doi.org/10.7589/JWD-D-21-00148}{10.7589/JWD-D-21-00148}

\item \textbf{McKee CD}\textsuperscript{*\dag}, Islam A\textsuperscript{\dag}, Rahman MZ, Khan SU, Rahman M, Satter SM, Islam A, Yinda CK, Epstein JH, Daszak P, Munster VJ. Nipah virus detection at bat roosts after spillover events, Bangladesh, 2012–2019. \textit{Emerging Infectious Diseases}. 2022; 28(7): 1384--1392. DOI: \href{https://doi.org/10.3201/eid2807.212614}{10.3201/eid-2807.212614}

\item Rice BL\textsuperscript{\dag}, Lessler J\textsuperscript{\dag}, \textbf{McKee C}\textsuperscript{\dag}, Metcalf CJE\textsuperscript{\dag}. Why do some coronaviruses become pandemic threats when others do not? \textit{PLOS Biology}. 2022; 20(5): e3001652. DOI: \href{https://doi.org/10.1371/journal.pbio.3001652}{10.1371/journal.pbio.3001652}

\item Ruiz-Aravena M\textsuperscript{\dag}, \textbf{McKee C}\textsuperscript{\dag}, Gamble A, Lunn T, Morris A, Snedden CE, Yinda CK, Port JR, Buchholz DW, Yeo YY, Faust C, …, Munster VJ, Plowright RK. Ecology, evolution and spillover of coronaviruses from bats. \textit{Nature Reviews Microbiology}. 2022; 20: 299--314. DOI: \href{https://doi.org/10.1038/s41579-021-00652-2}{10.1038/s41579-021-00652-2}

\item Redd AD, Peetluk LS, Jarrett BA, Hanrahan C, Schwartz S, Rao A, Jaffe AE, Peer AD, Jones CB, Lutz CS, \textbf{McKee CD}, …, Grabowski MK, Gurley ES, the Novel Coronavirus Research Compendium Team. Curating the evidence about COVID-19 for frontline public health and clinical care: the Novel Coronavirus Research Compendium. \textit{Public Health Reports}. 2022; 137(2): 197--202. DOI: \href{https://doi.org/10.1177/00333549211058732}{10.1177/00333549211058732}

\item Islam A, \textbf{McKee C}, Ghosh PK, Abedin J, Epstein JH, Daszak P, Luby SP, Khan SU, Gurley ES. Seasonality of date palm sap feeding behavior by bats in Bangladesh. \textit{EcoHealth}. 2021; 18: 359--371. DOI: \href{https://doi.org/10.1007/s10393-021-01561-9}{10.1007/s10393-021-01561-9}

\item Zorrilla VO, Lozano ME, Espada LJ, Kosoy M, \textbf{McKee C}, Valdivia HO, Arevalo H, Troyes M, Stoops CA, Fisher ML, Vásquez GM. Comparison of sand fly trapping approaches for vector surveillance of \textit{Leishmania} and \textit{Bartonella} species in ecologically distinct, endemic regions of Peru. \textit{PLOS Neglected Tropical Diseases}. 2021; 15(7): e0009517. DOI: \href{https://doi.org/10.1371/journal.pntd.0009517}{10.1371/journal.pntd.0009517}

\item \textbf{McKee CD}\textsuperscript{*}, Islam A, Luby SP, Salje H, Hudson PJ, Plowright RK, Gurley ES. The ecology of Nipah virus in Bangladesh: a nexus of land-use change and opportunistic feeding behavior in bats. \textit{Viruses}. 2021; 13(2): 169. DOI: \href{https://doi.org/10.3390/v13020169}{10.3390/v13020169}

\item \textbf{McKee C}\textsuperscript{*}, Bai Y, Webb C, Kosoy M. Bats are key hosts in the radiation of mammal-associated \textit{Bartonella} bacteria. \textit{Infection, Genetics and Evolution}. 2021; 89: 104719. DOI: \href{https://doi.org/10.1016/j.meegid.2021.104719}{10.1016/j.meegid.2021.104719}

\item Goodrich I, \textbf{McKee C}, Kosoy M. \textit{Trypanosoma} (\textit{Herpetosoma}) diversity in rodents and lagomorphs of New Mexico with a focus on epizootological aspects of infection in Southern Plains woodrats (\textit{Neotoma micropus}). \textit{PLOS ONE}. 2020; 15(12): e0244803. DOI: \href{https://doi.org/10.1371/journal.pone.0244803}{10.1371/journal.pone.-0244803}

\item Rudolf I, Blažejová H, Mendel J, Straková P, Šebesta O, Rettich F, Čabanová V, Miterpáková M, Betášová L, Peško J, Barbušinová E, \textbf{McKee C}, Osikowicz L, Šikutová S, Hubálek Z, Kosoy M. \textit{Bartonella} species in medically important mosquitoes, Central Europe. \textit{Parasitology Research}. 2020; 119(8): 2713--2717. DOI: \href{https://doi.org/10.1007/s00436-020-06732-1}{10.1007/s00436-020-06732-1}

\item Goodrich I, \textbf{McKee C}, Kosoy M. Longitudinal study of bacterial infectious agents in a community of small mammals in New Mexico. \textit{Vector-Borne and Zoonotic Diseases}. 2020; 20(7): 496--508. DOI: \href{https://doi.org/10.1089/vbz.2019.2550}{10.1089/vbz.2019.2550}

\item \textbf{McKee CD}\textsuperscript{*}, Krawczyk AI, Sándor AD, Görföl T, Földvári M, Földvári G, Dekeukeleire D, Haarsma A-J, Kosoy MY, Webb CT, Sprong H. Host phylogeny, geographic overlap, and roost sharing shape parasite communities in European bats. \textit{Frontiers in Ecology and Evolution}. 2019; 7: 69. DOI: \href{https://doi.org/10.3389/fevo.2019.00069}{10.3389/fevo.2019.00069}

\item Bai Y, Osinubi MOV, Osikowicz L, \textbf{McKee C}, Vora NM, Rizzo MR, Recuenco S, Davis L, Niezgoda M, Ehimiyein AM, Kia GSN, Oyemakinde A, Adeniyi OS, Gbadegesin YH, Saliman OA, Ogunniyi A, Ogunkoya AB, Kosoy MY, Idanre Bat Festival Investigation Team. Human exposure to novel \textit{Bartonella} species from contact with fruit bats. \textit{Emerging Infectious Diseases}. 2018; 24(12): 2317--2323. DOI: \href{https://doi.org/10.3201/eid2412.181204}{10.3201/eid2412.181204}

\item Kellner A, Carver S, Scorza V, \textbf{McKee CD}, Lappin M, Crooks KR, VandeWoude S, Antolin MF. Transmission pathways and spillover of an erythrocytic bacterial pathogen from domestic cats to wild felids. \textit{Ecology and Evolution}. 2018; 8(19): 9779--9792. DOI: \href{https://doi.org/10.1002/ece3.4451}{10.1002/ece3.4451}

\item \textbf{McKee CD}\textsuperscript{*}, Osikowicz LM, Schwedhelm TR, Maes SE, Enscore RE, Gage KL, Kosoy MY. Acquisition of \textit{Bartonella elizabethae} by experimentally exposed oriental rat fleas (\textit{Xenopsylla cheopis}; Siphonaptera, Pulicidae) and excretion of \textit{Bartonella} DNA in flea feces. \textit{Journal of Medical Entomology} 2018; 55(5): 1292--1298. DOI: \href{https://doi.org/10.1093/jme/tjy085}{10.1093/jme/tjy085}

\item Kosoy M\textsuperscript{\dag}, \textbf{McKee C}\textsuperscript{\dag}, Albayrak L, Fofanov Y. Genotyping of \textit{Bartonella} bacteria and their animal hosts: current status and perspectives. \textit{Parasitology}. 2018; 145(5): 543--562. DOI: \href{https://doi.org/10.1017/S0031182017001263}{10.1017/S0031182017001263}

\item Gorsich EE, \textbf{McKee CD}, Grear DA, Miller RS, Portacci K, Lindström T, Webb CT. Model-guided suggestions for targeted surveillance based on cattle shipments in the U.S. \textit{Preventive Veterinary Medicine}. 2018; 150: 52--59. DOI: \href{https://doi.org/10.1016/j.prevetmed.2017.12.004}{10.1016/j.prevetmed.2017.12.004}

\item \textbf{McKee CD}\textsuperscript{*}, Osikowicz LM, Schwedhelm TR, Bai Y, Castle KT, Kosoy MY. Survey of parasitic bacteria in bat bugs, Colorado. \textit{Journal of Medical Entomology.} 2018; 55(1): 237--241. DOI: \href{https://doi.org/10.1093/jme/tjx155}{10.1093/jme/tjx155}

\item \textbf{McKee CD}\textsuperscript{*}, Kosoy MY, Bai Y, Osikowicz LM, Franka R, Gilbert AT, Boonmar S, Rupprecht CE, Peruski LF. Diversity and phylogenetic relationships among \textit{Bartonella} strains from Thai bats. \textit{PLOS ONE}. 2017; 12(7): e0181696. DOI: \href{https://doi.org/10.1371/journal.pone.0181696}{10.1371/journal.pone.0181696}

\item Urushadze L, Bai Y, Osikowicz L, \textbf{McKee C}, Sidamonidze K, Putkaradze D, Imnadze P, Kandaurov A, Kuzmin I, Kosoy M. Prevalence, diversity, and host associations of \textit{Bartonella} strains in bats from Georgia (Caucasus). \textit{PLOS Neglected Tropical Diseases}. 2017; 11(4): e0005428. DOI: \href{https://doi.org/10.1371/journal.pntd.0005428}{10.1371/journal.pntd.0005428}

\item Bai Y, Urushadze L, Osikowicz L, \textbf{McKee C}, Kuzmin I, Kandaurov A, Babuadze G, Natradze I, Imnadze P, Kosoy M. Molecular survey of bacterial zoonotic agents in bats from the country of Georgia (Caucasus). \textit{PLOS ONE}. 2017; 12(1): e0171175. DOI: \href{https://doi.org/10.1371/journal.pone.0171175}{10.1371/journal.pone.0171175}

\item \textbf{McKee CD}\textsuperscript{*}, Hayman DTS, Kosoy MY, Webb CT. Phylogenetic and geographic patterns of bartonella host shifts among bat species. \textit{Infection, Genetics and Evolution}. 2016; 44: 382--394. DOI: \href{https://doi.org/10.1016/j.meegid.2016.07.033}{10.1016/j.meegid.2016.07.033}

\item Zinzow-Kramer WM, Horton BM, \textbf{McKee CD}, Michaud JM, Tharp GK, Thomas JW, Tuttle EM, Yi S, Maney DL. Genes located in a chromosomal inversion are correlated with territorial song in white-throated sparrows. \textit{Genes, Brain and Behavior}. 2015; 14(8): 641--654. DOI: \href{https://doi.org/10.1111/gbb.12252}{10.1111/gbb.12252}

\item Bai Y, Hayman DTS, \textbf{McKee CD}, Kosoy MY. Classification of \textit{Bartonella} strains associated with straw-colored fruit bats (\textit{Eidolon helvum}) across Africa using a multi-locus sequence typing platform. \textit{PLOS Neglected Tropical Diseases}. 2015; 9(1): e0003478. DOI: \href{https://doi.org/10.1371/journal.pntd.0003478}{10.1371/journal.pntd.0003478}

\end{pubenum}

\subsection*{\textit{Preprints}}

\begin{pubenum}

\item Stevens T, Zimmerman R, Albery G, Becker DJ, Kading R, Keiser CN, Khandelwal S, Kramer-Schadt S, Krut-Landau R, \textbf{McKee C}, Montecino-Latorre D, …, Carlson CJ. A minimum data standard for wildlife disease studies. \textit{EcoEvoRxiv. 2024}. DOI: \href{https://doi.org/10.32942/x2tw4j}{10.32942/x2tw4j}.

\item Jackson J\textsuperscript{\ddag}, Shanta IS, \textbf{McKee C}, Luby SP, Haider N, Sharker Y, Plowright R, Hudson P, Gurley E. Identifying weather patterns affecting household date palm sap consumption in Bangladesh, 2013-2016. \textit{medRxiv. 2024}. DOI: \href{https://doi.org/10.1101/2024.05.06.24306951}{10.1101/2024.05.06.24306951}.

\item \textbf{McKee CD}\textsuperscript{*}, Webb CT, Kosoy MY, Bai Y, Osikowicz LM, Suu-Ire R, Ntiamoa-Baidu Y, Cunningham AA, Wood JL, Hayman DT. Manipulating vector transmission reveals local processes in bacterial communities of bats. \textit{bioRxiv. 2021}. DOI: \href{https://doi.org/10.1101/2021.03.03.433743}{10.1101/2021.03.03.433743}

 \end{pubenum}

% \subsection*{\textit{Manuscripts Submitted or in Preparation}}

% \begin{pubenum}

% \item Carrazco-Montalvo A, Zorrilla VO, Espada LJ, Fárez-Noblecilla LA, Lozano ME, Kosoy M, \textbf{McKee C}, Stoops CA, Larson R, Leon R, Vásquez GM. Phlebotomine fauna characterization and \textit{Bartonella bacilliformis} natural infection in \textit{Pintomyia robusta} at the Ecuador-Peru frontier. \textit{In advanced preparation for Nature Communications}.

% \end{pubenum}

%----------------------------------------------------------------------------------------
%	HONORS AND AWARDS
%----------------------------------------------------------------------------------------

\section*{HONORS AND AWARDS}

Excellence in Teaching Award $\cdot$ Johns Hopkins Bloomberg School of Public Health (2023 \& 2024)

Vice President for Research Fellowship $\cdot$ Colorado State University (2018) %\$4000

Department of Biology Travel Awards $\cdot$ Colorado State University (2014 \& 2017) %\$2000/year

Sharon E. and David E. Kabes Scholarship $\cdot$ Colorado State University (2015) %\$1150,

GDPE Research Grant $\cdot$ Colorado State University (2014) %\$2000

Department of Biology Graduate Fellowship $\cdot$ Colorado State University (2013) %\$1000

Phi Beta Kappa $\cdot$ University of Pittsburgh (2011)

% Honors College Scholarship $\cdot$ University of Pittsburgh (2007) % $4000

%----------------------------------------------------------------------------------------
%	PRESENTATIONS
%----------------------------------------------------------------------------------------

\section*{PRESENTATIONS}

\subsection*{Invited Talks}

\href{https://www.youtube.com/live/MOgTl1EUSes?si=iMqpRzL8OlzRsdlE}{Environmental and behavioral drivers of cross-species Nipah virus transmission in Bangladesh} $\cdot$ 
Emerging Infectious Diseases: Ecology and Evolution workshop $\cdot$ International Centre for Theoretical Sciences $\cdot$ Bengaluru, India (2024)

\href{https://www.youtube.com/live/WlhZYbyfGiw?si=1AW1gtXbbL0bbaPp}{Ecological stressors and pathogen shedding} $\cdot$ Emerging Infectious Diseases: Ecology and Evolution workshop $\cdot$ International Centre for Theoretical Sciences $\cdot$ Bengaluru, India (2024)

Investigating the ecology of Nipah and other bat-borne viruses at the human-animal interface in Bangladesh $\cdot$ Uniformed Services University of the Health Sciences $\cdot$ Bethesda, MD (2024)

Investigating Nipah virus spillover at the human-animal interface in Bangladesh $\cdot$ Cornell University College of Veterinary Medicine $\cdot$ Ithaca, NY (2023)

Global change, infectious disease, and public health: the need for ecological interventions $\cdot$ WHO GOARN/RCCE Collective Service $\cdot$ Virtual (2022)

\subsection*{Campus or Departmental Talks}

Investigating the ecology of Nipah and other bat-borne viruses at the human-animal interface in Bangladesh $\cdot$ Johns Hopkins Bloomberg School of Public Health $\cdot$ Baltimore, MD (2023)

\href{https://www.youtube.com/watch?v=lYWR_dZRdjE}{Bad roommates? Inferring bacterial interactions in coinfected individuals} $\cdot$ Colorado State University Vice President Office of Research Symposium $\cdot$ Fort Collins, CO (2018)

\subsection*{Scientific Meetings}

\subsubsection*{Oral Presentations}

Empowering researchers through Publish-Review-Curate workflows $\cdot$ Society for Scholarly Publishing $\cdot$ Boston, MA (2024)

Superspreading of SARS-CoV-2: a systematic review and meta-analysis $\cdot$ American Society of Tropical Medicine and Hygiene $\cdot$ Chicago, IL (2023)

Comprehensive time tree analysis identifies bats as key to the radiation of mammal-associated \textit{Bartonella} bacteria $\cdot$ Evolution $\cdot$ Providence, RI (2019)

Timing the diversification of a mammal parasite, \textit{Bartonella} $\cdot$ Ecological Society of America $\cdot$ New Orleans, LA (2018)

Timing the diversification of a mammal parasite, \textit{Bartonella} $\cdot$ Front Range Student Ecology Symposium $\cdot$ Fort Collins, CO (2018)

Long-term monitoring of \textit{Bartonella} spp. bacteria in a captive colony of fruit bats and experimental evidence of bat flies as vectors of bartonella $\cdot$ Ecological Society of America $\cdot$ Portland, OR (2017)

Long-term monitoring of \textit{Bartonella} spp. bacteria in a captive colony of fruit bats and experimental evidence of bat flies as vectors of bartonella $\cdot$ International Symposium on Infectious Diseases of Bats $\cdot$ Fort Collins, CO (2017)

Host phylogenetic distance and ectoparasite overlap predict \textit{Bartonella} sharing in European bats $\cdot$ Front Range Student Ecology Symposium $\cdot$ Fort Collins, CO (2017)

Phylogeography of \textit{Bartonella} bacteria in \textit{Eidolon} spp. fruit bats across Africa $\cdot$ Ecological Society of America $\cdot$ Baltimore, MD (2015)

Phylogeography of \textit{Bartonella} in \textit{Eidolon} fruit bats across Africa $\cdot$ Front Range Student Ecology Symposium $\cdot$ Fort Collins, CO (2015)

\subsubsection*{Poster Presentations}

Nipah virus detection at bat roosts following spillover events in Bangladesh, 2012–2019 $\cdot$ International Conference on Emerging Infectious Diseases $\cdot$ Atlanta, GA (2022)

Manipulating vector transmission reveals local processes in bacterial communities of bats $\cdot$ Ecology and Evolution of Infectious Disease $\cdot$ Princeton, NJ (2019) $\cdot$ \textbf{Poster award}

Dispersal of hosts and vectors predicts bacterial community structure across an island chain $\cdot$ Front Range Student Ecology Symposium $\cdot$ Fort Collins, CO (2019) $\cdot$ \textbf{Poster award}

Timing the diversification of a mammal parasite, \textit{Bartonella} $\cdot$ Colorado State University Graduate Student Showcase $\cdot$ Fort Collins, CO (2017)

Linking patterns of bacterial parasite diversity across host and vector communities $\cdot$ Ecology and Evolution of Infectious Disease $\cdot$ Santa Barbara, CA (2017)

Phylogeography of \textit{Bartonella} bacteria in \textit{Eidolon} spp. fruit bats across Africa $\cdot$ Ecology and Evolution of Infectious Disease $\cdot$ Athens, GA (2015)

Large-scale patterns of \textit{Bartonella} prevalence and diversity in African fruit bats $\cdot$ Ecological Society of America $\cdot$ Sacramento, CA (2014)

Large-scale patterns of \textit{Bartonella} prevalence and diversity in African fruit bats $\cdot$ Ecology and Evolution of Infectious Disease $\cdot$ Fort Collins, CO (2014)

A novel method of isolating multi-locus sequence data for characterizing \textit{Bartonella} diversity in bats $\cdot$ Front Range Student Ecology Symposium $\cdot$ Fort Collins, CO (2014)

%----------------------------------------------------------------------------------------
%	EDITORIAL AND OTHER PEER REVIEW ACTIVITIES
%----------------------------------------------------------------------------------------

\section*{EDITORIAL AND OTHER PEER REVIEW ACTIVITIES}

\textbf{Journal Peer Review}: Acta Chiropterologica $\cdot$ Acta Tropica $\cdot$ BMC Microbiology $\cdot$ BMJ $\cdot$ Current Zoology $\cdot$ Ecology and Evolution $\cdot$ Frontiers in Microbiology $\cdot$ Frontiers in Veterinary Science $\cdot$ Infection, Genetics and Evolution $\cdot$ Journal of Animal Ecology $\cdot$ Journal of Medical Entomology $\cdot$ Journal of Zoological Systematics and Evolutionary Research $\cdot$ Microbial Ecology $\cdot$ Microbial Pathogenesis $\cdot$ Molecular Ecology $\cdot$ mSphere $\cdot$ Nature Microbiology $\cdot$ Parasitology Research $\cdot$ Pathogens and Global Health $\cdot$ PLOS Neglected Tropical Diseases $\cdot$ PLOS ONE $\cdot$ The European Zoological Journal $\cdot$ The Lancet Planetary Health $\cdot$ The Science of Nature $\cdot$ Scientific Reports $\cdot$ Systematic Biology $\cdot$ Tropical Biomedicine $\cdot$ Veterinary Microbiology $\cdot$ Viruses $\cdot$ Zoonoses and Public Health

\textbf{Research Grant Review}: Netherlands Organization for Health Research and Development (ZonMw), Infectious Disease Control Program

%  \section*{ACADEMIC SERVICE}

% Executive Committee student member, Graduate Degree Program in Ecology, Colorado State University (7/1/15-6/30/16)

% Faculty search committee student organizer, Department of Biology, Colorado State University (2015)

%----------------------------------------------------------------------------------------
%	PROFESSIONAL ACTIVITIES
%----------------------------------------------------------------------------------------

\section*{PROFESSIONAL ACTIVITIES}

\subsection*{Academic Service}

External Examiner for PhD viva exam $\cdot$ Laura MacKenzie, School of Biological Sciences, University of Aberdeen (2023)

Organizer $\cdot$ JHSPH Infectious Disease Dynamics Group Meeting (2021--2023)

Vice President $\cdot$ Front Range Student Ecology Symposium (2014--2015)

Volunteer  $\cdot$ Front Range Student Ecology Symposium (2014--2019)

\subsection*{Community Service}

Exam Writer $\cdot$ Disease Detectives, Maryland Science Olympiad (2021--2022)

Moderator $\cdot$ Rocky Mountain Regional Middle School Science Bowl (2018 \& 2019)

Judge $\cdot$ O'Dea Elementary School Science Fair (2015)

\subsection*{Society Memberships}

Member $\cdot$ North American Society for Bat Research (2021--present)

Member $\cdot$ American Society of Tropical Medicine and Hygiene (2021--present)

Member $\cdot$ Society for the Study of Evolution (2019 \& 2021)

Member $\cdot$ American Society of Naturalists (2019)

Member \& Volunteer $\cdot$ Ecological Society of America (2014--2019)

%----------------------------------------------------------------------------------------
%	PRACTICE ACTIVITIES
%----------------------------------------------------------------------------------------

\section*{PRACTICE ACTIVITIES}

% \subsection*{Practice-Related Reports}

\subsection*{Presentations to Policymakers, Communities, and Other Stakeholders}

Bat Virus Spillover Virus Compendium (Bat-Com) $\cdot$ Led team of research to curate and write public-facing summaries and assessments of research articles on the spillover of SARS-CoV, SARS-CoV-2, and related coronaviruses

\href{https://ncrc.jhsph.edu/}{2019 Novel Coronavirus Research Compendium (NCRC)} $\cdot$ Curated and wrote public-facing summaries and assessments of research articles on the Ecology \& Spillover of SARS-CoV-2 and related coronaviruses (2020--2021)

\href{https://usamm-gen-net.shinyapps.io/usamm-gen-net/}{USAMM R Shiny Visualization} $\cdot$ Developed tool for visualizing estimated county-to-county animal shipments in the US $\cdot$ Presented to stakeholders at USDA APHIS (2015)

\subsection*{Media Coverage}

\href{https://thispodcastwillkillyou.com/2024/05/21/episode-140-nipah-virus-of-fruit-and-bats/}{Nipah virus: Of Fruit and Bats} $\cdot$ \href{https://open.spotify.com/episode/3bF2bICaa7Ipq7s6EzCUKj?si=456f659ff8874de5}{Interview} with \textit{This Podcast Will Kill You} (2024)

\href{https://youtu.be/p6I0kRVGyo4}{Disease on the Wing} $\cdot$ YouTube video produced by \textit{The Scientist Magazine} (2014)

\href{https://www.the-scientist.com/features/lurking-in-the-shadows-36302}{Lurking in the Shadows} $\cdot$ Feature article in \textit{The Scientist Magazine} (2014)

% \subsection*{Media Dissemination}

%----------------------------------------------------------------------------------------
%	SOFTWARE
%----------------------------------------------------------------------------------------

\section*{SOFTWARE}

% \subsection*{GitHub Templates}

% DaSL Collection - \textit{An automatically generating resource documenting all completed Data Science Lab content.} [\href{https://github.com/fhdsl/DaSL_Collection}{Available on GitHub}]

% Fred Hutch Letterhead - \textit{A LaTeX template for Fred Hutch-themed letterhead.} [\href{https://github.com/fhdsl/FH_letterhead}{Available on GitHub}]

% Online Tools for Training Resources (OTTR) Template - \textit{A GitHub template that simplifies and accelerates publishing course content in bookdown format or to Leanpub and Coursera. Created with Candace Savonen, Carrie Wright, and others.} [\href{https://github.com/jhudsl/OTTR_Template}{Available on GitHub}]

% AnVIL Template - \textit{A GitHub template variation of the OTTR Template that automatically formats and generates content specific to the AnVIL Project. Created with Katherine Cox.} [\href{https://github.com/jhudsl/AnVIL_Template}{Available on GitHub}]

\subsection*{Software \& Tools}

The Flexible Epidemic Modeling Pipeline (\href{https://www.flepimop.org/}{\textit{flepiMoP}}) $\cdot$ A software suite for simulating a wide range of compartmental models of infectious disease transmission [\href{https://github.com/HopkinsIDD/flepiMoP}{Available on GitHub}]

%----------------------------------------------------------------------------------------
%	RESEARCH EXPERIENCE
%----------------------------------------------------------------------------------------

\section*{RESEARCH EXPERIENCE}

Graduate Research Assistant $\cdot$ Colorado State University $\cdot$ Fort Collins, CO (2015 \& 2019) \\
Department of Biology $\cdot$ Supervisor: Prof. Colleen Webb \\
\textit{Data analysis and modeling of livestock movement and pathogen spread}

Guest Researcher $\cdot$ CDC Division of Vector-Borne Diseases $\cdot$ Fort Collins, CO (2014--2019) \\
Bacterial Diseases Branch $\cdot$ Supervisor: Dr. Michael Kosoy \\
\textit{Molecular detection and phylogenetic analysis of bacterial pathogens in wildlife; experimental infection of rodents and fleas with bacterial pathogens}

Field Assistant \& Field Research Specialist $\cdot$ Emory University $\cdot$ Atlanta, GA (2010 \& 2013) \\
Department of Psychology $\cdot$ Supervisors: Prof. Donna Maney, Prof. Brent Horton \\
\textit{Behavioral assays and tissue sampling of wild birds in Maine}

Laboratory Technician $\cdot$ W.L. Gore \& Associates $\cdot$ Newark, DE (2011--2013) \\
Industrial Products Division, New Product Development \\
\textit{Prototype development and performance testing for new filtration products}

Field Technician $\cdot$ University of California, Santa Cruz $\cdot$ Santa Cruz, CA (2011) \\
Department of Ecology and Evolutionary Biology $\cdot$ Supervisor: Prof. Marm Kilpatrick \\
\textit{Blood sampling from wild birds for West Nile virus surveillance in Washington, DC area}

%----------------------------------------------------------------------------------------
% Break page and add Part II header

\newpage
\name{CLIFTON D. MCKEE}
\parttwo

%----------------------------------------------------------------------------------------

%----------------------------------------------------------------------------------------
%	TEACHING
%----------------------------------------------------------------------------------------

\section*{TEACHING}

% \subsection*{Capstone Advisees}

\subsection*{Research Advisees}

% \subsubsection*{Postdoctoral Fellows}

\subsubsection*{Master's Students}

Chan, Elias $\cdot$ Master of Health Science, Epidemiology $\cdot$ Johns Hopkins Bloomberg School of Public Health (2024 -- present)

Niu, Yannan $\cdot$ Master of Health Science, Epidemiology $\cdot$ Johns Hopkins Bloomberg School of Public Health (2023 -- 2024) $\cdot$ Thesis: Bat viral shedding: a review of seasonal patterns and risk factors (\textit{manuscript submitted to Vector-Borne and Zoonotic Diseases})

Garcia, Andrés $\cdot$ Master of Health Science, Epidemiology $\cdot$ Johns Hopkins Bloomberg School of Public Health (2023 -- 2024) $\cdot$ Thesis: Literature review on risk factors for transmission of \textit{Orientia tsutsugamushi} in Thailand, Vietnam, and Malaysia

Jackson, Jules $\cdot$ Master of Science, Epidemiology $\cdot$ Johns Hopkins Bloomberg School of Public Health (2022--2023) $\cdot$ Thesis: Identifying weather patterns affecting household date palm sap consumption in Bangladesh, 2013-2016 (\textit{manuscript submitted to PLOS ONE})

Endres, Kelly $\cdot$ Master of Science in Public Health, Research Assistant $\cdot$ Johns Hopkins Bloomberg School of Public Health (2021--2022)

\subsubsection*{Undergraduate Students}

Lazarus, Paul $\cdot$ Research Assistant $\cdot$ Colorado State University (2018) \\
Cannella, Alexis $\cdot$ Research Assistant $\cdot$ Colorado State University (2016--2017) \\
Doyle, Jake $\cdot$ Research Assistant $\cdot$ Colorado State University (2014--2015) \\
Leming, Madison $\cdot$ Research Assistant $\cdot$ Colorado State University (2014--2015)

\subsection*{Classroom Instruction -- Instructor of Record}

\subsubsection*{Johns Hopkins Bloomberg School of Public Health}

\href{https://jhudatascience.org/intro_to_r/}{Introduction to \texttt{R} for Public Health Researchers} (140.604.79) $\cdot$ Summer Institute 2024 (enrollment: 16)

The One Health Approach to Epidemiology and Global Public Health (340.610.81) $\cdot$ Spring 4th Term 2024 (enrollment: 44)

\href{https://jhudatascience.org/intro_to_r/}{Introduction to \texttt{R} for Public Health Researchers} (140.604.73) $\cdot$ Winter Institute 2024 (enrollment: 27) $\cdot$ \textbf{Excellence in Teaching award}

\href{https://jhudatascience.org/intro_to_r/}{Introduction to \texttt{R} for Public Health Researchers} (140.604.79) $\cdot$ Summer Institute 2023 (enrollment: 35) $\cdot$ \textbf{Excellence in Teaching award}

The One Health Approach to Epidemiology and Global Public Health (340.610.81) $\cdot$ Spring 4th Term 2023 (enrollment: 35)

\subsection*{Classroom Instruction -- Teaching Assistant}

\subsubsection*{Johns Hopkins Bloomberg School of Public Health}

Topics in Infectious Disease Epidemiology (340.668.89) $\cdot$ Teaching Assistant (2022)

The One Health Approach to Epidemiology and Global Public Health (340.610.81) $\cdot$ Teaching Assistant (2021 \& 2022)

Epidemiology of Infectious Diseases (340.627.01) $\cdot$ Teaching Assistant (2020 \& 2021)

\subsubsection*{Colorado State University}

Molecular and General Genetics (BZ 350) $\cdot$ Teaching Assistant \& Recitation Instructor (2014, 2018 \& 2019)

Introduction to Evolution (BZ 220) $\cdot$ Teaching Assistant (2018)

Ecology (LIFE 320) $\cdot$ Teaching Assistant (2017)

Ornithology (BZ 335) $\cdot$ Teaching Assistant \& Laboratory Instructor (2014 \& 2015)

Attributes of Living Systems (LIFE 102) $\cdot$ Teaching Assistant \& Laboratory Instructor (2013)

\subsection*{Online Courses}

Coursera $\cdot$ Developing content for the Ecology and Environment module in an upcoming course on One Health Investigations of Outbreaks and Spillovers (2024)

\subsection*{Short Courses \& Workshops}

Coordinator \& Content Developer $\cdot$ Applied Modeling in Public Health workshop $\cdot$ Johns Hopkins Bloomberg School of Public Health (2022 -- present)

Content Developer \& Co-Instructor $\cdot$ Zombiecology: Workshop on Disease Ecology $\cdot$ National Science Olympiad $\cdot$ Colorado State University (2018)

% \subsection*{Other Teaching}

% Evolution \& medicine: how evolution influences diseases \& our response to them $\cdot$ Guest lecture in Introduction to Evolution (BZ 220) $\cdot$ Colorado State University (2018)

% Phylogenetic analysis of pathogens $\cdot$ Guest lecture in Principles of Systematic Zoology (BZ 424) $\cdot$ Colorado State University (2018)

% Vampires, rabies, and bugs -- Oh my! $\cdot$ Guest lecture in Tropical Ecology and Evolution $\cdot$ Colorado State University (2017)

% Bats and their parasites $\cdot$ Guest lecture in Ecology of Infectious Diseases (BZ 418) $\cdot$ Colorado State University (2016)

% Prokaryotic genetics $\cdot$ Guest lecture in Molecular and General Genetics (BZ 350) $\cdot$ Colorado State University (2014)

%----------------------------------------------------------------------------------------
% Break page and add Part III header

\newpage
\name{CLIFTON D. MCKEE}
\partthree

%----------------------------------------------------------------------------------------

%----------------------------------------------------------------------------------------
%	RESEARCH GRANT PARTICIPATION
%----------------------------------------------------------------------------------------

\section*{RESEARCH GRANT PARTICIPATION}

\subsection*{Current Support}

Project Title: Atlantic Coast Center for Infectious Disease Dynamics and Analytics \\
Dates: 09/12/2023 -- 09/11/2028 \\
Sponsoring Agency: CDC Center for Forecasting and Outbreak Analytics \\
Principal Investigator: Shaun Truelove \\
Main Grant Objective: Lead recruitment and logistics for Applied Modeling in Public Health workshop series; assist with workshop content; participate in model development for forecasts and scenario projections of COVID-19, influenza, and RSV cases and hospitalizations \\
Role: Research Associate

Project Title: Ecology of Nipah Virus in Bangladesh (58-3022-3-029) \\
Dates: 09/01/2023 -- 08/31/2026 \\
Sponsoring Agency: USDA/ARS \\
Principal Investigator: Emily Gurley \\
Main Grant Objective: Provide support for field work, data analysis, and writing manuscripts; draft protocols for approval, collate and analyze data coming from the field and the laboratory, and lead development of manuscripts for publication \\
Role: Research Associate

Project Title: Safety and Healthcare Epidemiology Prevention Research Development (SHEPheRD) (75D30121F00005) \\
Dates: 09/03/2021 -- 09/02/2024 \\
Sponsoring Agency: Centers for Disease Control \\
Principal Investigator: Shaun Truelove \\
Main Grant Objective: Lead recruitment and logistics for Applied Modeling in Public Health workshop series; assist with workshop content; participate in model development for forecasts and scenario projections of COVID-19, influenza, and RSV cases and hospitalizations \\
Role: Research Associate

Project Title: Solving Opportunities for Spillover (SOS): Frequency and Mechanisms of Cross-species Transmission of Henipaviruses in Bangladesh (1U01AI168287-01A1) \\
Dates: 01/03/2023 -- 01/02/2028 \\
Sponsoring Agency: NIH/NIAID \\
Principal Investigator: Emily Gurley \\
Main Grant Objective: Provide support for field work, data analysis, and writing manuscripts; draft protocols for approval, collate and analyze data coming from the field and the laboratory, and lead development of manuscripts for publication \\
Role: Research Associate

\subsection*{Past Support}

Project Title: Preventing emergence and spillover of bat pathogens in high risk global hotspots (G166-19-W7329) \\
Dates: 10/01/2018 -- 09/30/2022 \\
Sponsoring Agency: DARPA/Montana State University \\
Principal Investigator: Raina Plowright (Project Director), Emily Gurley (Site PI) \\
Main Grant Objective: Work with the team in Bangladesh to develop and implement protocols to sample bats, including identifying roosts best suited for sampling, ensuring protocols are aligned with broad aims of this grant, and that work is delivered on time; real-time analysis of virus detection data \\
Role: Postdoctoral Fellow

Project Title: COVID-19 Technical Lead in the Implementation of the Strategic Preparedness (202548990) \\
Dates: 08/03/2022 -- 11/03/2022 \\
Sponsoring Agency: World Health Organization \\
Principal Investigator: Emily Gurley \\
Main Grant Objective: Support COVID-19 technical lead to improve evidence-based guidance by completing a systematic review of COVID-19 superspreading \\
Role: Postdoctoral Fellow

%----------------------------------------------------------------------------------------
%	ADDITIONAL INFORMATION 
%----------------------------------------------------------------------------------------

\section*{ADDITIONAL INFORMATION}

\subsection*{Personal Statement}

I am an infectious disease ecologist interested in zoonotic pathogens and their dynamics within host populations and at the human-animal interface. Overall, my research sits firmly within One Health, seeking to understand disease emergence at the intersection between public health, animal ecology, and environmental change. I use a combination of fieldwork, statistical modeling, molecular genetics, and phylogenetics to investigate pathogen persistence in animal reservoirs, evolution of host specificity, and ecological drivers of pathogen spillover.

\subsection*{Keywords}

disease ecology, epidemiology, data science, zoonoses, One Health, phylogenetics, evolution

\subsection*{Skills}

\textbf{Bioinformatics and Phylogenetics}: Multiple sequence alignment (CLUSTAL, MAFFT), maximum likelihood (MEGA, IQ-Tree, RAxML) and Bayesian (MrBayes) phylogenetic inference, inference of population genetic structure (STRUCTURE) and recombination (ClonalFrame, SplitsTree), cophylogeny (ParaFit, PACo), ancestral state reconstruction and phylodynamics (BEAST)

\textbf{Computing and Data Visualization}: High proficiency in R for data manipulation, statistical analysis, and visualization (base and tidyverse, ggplot2, igraph), generating reports with R Markdown, and interactive web pages with Shiny; experience with git, \LaTeX, and Python (NumPy, Pandas, seaborn)

\textbf{Modeling}: Proficient in deterministic and stochastic models for infectious disease dynamics and hierarchical Bayesian modeling (JAGS)

\end{document}