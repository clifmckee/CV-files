%%%%%%%%%%%%%%%%%%%%%%%%%%%%%%%%%%%%%%%
% Deedy CV/Resume
% XeLaTeX Template
% Version 1.0 (5/5/2014)
%
% This template has been downloaded from:
% http://www.LaTeXTemplates.com
%
% Original author:
% Debarghya Das (http://www.debarghyadas.com)
% With extensive modifications by:
% Vel (vel@latextemplates.com)
%
% License:
% CC BY-NC-SA 3.0 (http://creativecommons.org/licenses/by-nc-sa/3.0/)
%
% Important notes:
% This template needs to be compiled with XeLaTeX.
%
%%%%%%%%%%%%%%%%%%%%%%%%%%%%%%%%%%%%%%

\documentclass[letterpaper]{deedy-resume} % Use US Letter paper, change to a4paper for A4 

\usepackage{array}
\usepackage{xcolor}
\usepackage{marvosym}

\newcommand{\spread}{\vspace{1mm}}

\begin{document}
\fontsize{10pt}{14pt}\selectfont

%----------------------------------------------------------------------------------------
%	TITLE SECTION
%----------------------------------------------------------------------------------------

\lastupdated % Print the Last Updated text at the top right

\namesection{Clifton}{McKee}% Your name
{Disease Ecologist} %title
{
Department of Epidemiology, Bloomberg School of Public Health, Johns Hopkins University\\
615 N Wolfe St, Suite E6008 $\cdot$ Baltimore, MD 21205\\ 
\href{mailto:clifton.mckee@gmail.com}{\Letter~clifton.mckee@gmail.com} \href{tel:+19708899540}{\Mobilefone~+1 970 889 9540} \Mundus~% Your contact information
\urlstyle{same}\href{https://clifmckee.github.io/}{clifmckee.github.io}\\ % Your website, LinkedIn profile or other web address
}
\hfill

%----------------------------------------------------------------------------------------
%	INTERESTS
%----------------------------------------------------------------------------------------

\section{Interests}
\raggedright{wildlife disease ecology $\cdot$ epidemiology $\cdot$ phylogenetics $\cdot$ statistical modeling $\cdot$ population genetics $\cdot$ molecular biology}
\sectionspace

%----------------------------------------------------------------------------------------
%	EDUCATION
%----------------------------------------------------------------------------------------

\section{Education} 
\begin{tabular}{>{\raggedright\arraybackslash}p{2cm}p{16cm}}
2015–2020 & \textbf{PhD in Ecology} Colorado State University $\cdot$ Fort Collins, CO\\
& Advisor: Prof Colleen Webb \\
2013–2015 & \textbf{MS in Ecology} Colorado State University $\cdot$ Fort Collins, CO\\
& Advisor: Prof Colleen Webb \\
2007–2011 & \textbf{BS in Ecology} \& \textbf{BA in Environmental Studies} University of Pittsburgh $\cdot$ Pittsburgh, PA\\
\end{tabular}
\sectionspace

%----------------------------------------------------------------------------------------
%	PROFESSIONAL EXPERIENCE
%----------------------------------------------------------------------------------------

\section{Professional Experience}
\begin{tabular}{>{\raggedright\arraybackslash}p{2cm}p{16cm}}
2020– & \textbf{Johns Hopkins University} Postdoctoral Fellow $\cdot$ Baltimore, MD\\
& Supervisor: Dr Emily Gurley \\
2015 \& 2019 & \textbf{Colorado State University} Research Assistant $\cdot$ Fort Collins, CO\\
& Supervisor: Prof Colleen Webb \\
2014–2019 & \textbf{CDC Division of Vector-Borne Diseases} Guest Researcher $\cdot$ Fort Collins, CO\\
& Supervisor: Dr Michael Kosoy \\
2015–2017 & \textbf{CDC Division of Vector-Borne Diseases} Regular Fellow $\cdot$ Fort Collins, CO\\
& Supervisor: Dr Michael Kosoy \\
2010 \& 2013 & \textbf{Emory University} Field Assistant \& Field Research Specialist $\cdot$ Atlanta, GA\\
& PI: Prof Donna Maney \\
2011–2013 & \textbf{WL Gore \& Associates} Laboratory Technician $\cdot$ Newark, DE \\
2011 & \textbf{University of California, Santa Cruz} Field Technician $\cdot$ Santa Cruz, CA \\
& PI: Prof Marm Kilpatrick \\
\end{tabular}
\sectionspace

%------------------------------------------------
% PUBLICATIONS
%------------------------------------------------

\section{Publications}

\textsuperscript{\dag}Equal contribution\\

\sectionspace

\begin{tabular}{>{\raggedright\arraybackslash}p{2cm}p{16cm}}

2022 & Goodrich I, \textbf{McKee C}, Margos G, Kosoy M. Molecular characterization of a novel relapsing fever \textit{Borrelia} species from the desert cottontail (\textit{Sylvilagus audubonii}) in New Mexico, USA.  In press at \href{https://meridian.allenpress.com/jwd/article-abstract/doi/10.7589/JWD-D-21-00148/483308/Molecular-characterization-of-a-novel-relapsing}{\textcolor{special}{Journal of Wildlife Diseases}.} \\

2022 & \textbf{McKee CD,} Islam A, Rahman MZ, Khan SU, Rahman M, Satter SM, et al. Nipah virus detection at bat roosts after spillover events, Bangladesh, 2012–2019. \href{https://doi.org/10.3201/eid2807.212614}{\textcolor{special}{Emerging Infectious Diseases. 2022;28(7):1384-1392}.} \\

2022 & Rice BL\textsuperscript{\dag}, Lessler J\textsuperscript{\dag}, \textbf{McKee C}\textsuperscript{\dag}, Metcalf CJE\textsuperscript{\dag}. Why do some coronaviruses become pandemic threats when others do not? \href{https://doi.org/10.1371/journal.pbio.3001652}{\textcolor{special}{PLoS Biology. 2022;20(5):e3001652}. } \\

2022 & Ruiz-Aravena M\textsuperscript{\dag}, \textbf{McKee C}\textsuperscript{\dag}, Gamble A, Lunn T, Morris A, Snedden CE, Yinda CK, Port JR, Buchholz DW, Yeo YY, Faust C, Jax E, Dee L, Jones DN, Kessler MK, Falvo C, Crowley D, Bharti N, Brook CE,  Aguilar HC, Peel AJ, Restif O, Schountz T, Parrish CR, Gurley ES, Lloyd-Smith JO, Hudson PJ, Munster VJ, Plowright RK. Ecology, evolution and spillover of coronaviruses from bats. \href{https://doi.org/10.1038/s41579-021-00652-2}{\textcolor{special}{Nature Reviews Microbiology. 2022;20:299–314}.} \\

% FORCE PAGE BREAK
\end{tabular}
\section{Publications (continued)}
\begin{tabular}{>{\raggedright\arraybackslash}p{2cm}p{16cm}}

2022 & Redd AD, Peetluk LS, Jarrett BA, Hanrahan C, Schwartz S, Rao A, Jaffe AE, Peer AD, Jones CB, Lutz CS, \textbf{McKee CD}, Patel EU, Rosen JG,  Garrison Desany H, McKay HS, Muschelli J, Andersen KM, Link MA, Wada N, Baral P, Young R, Boon D, Grabowski MK, Gurley ES, the Novel Coronavirus Research Compendium Team. Curating the evidence about COVID-19 for frontline public health and clinical care: the Novel Coronavirus Research Compendium. \href{https://doi.org/10.1177/00333549211058732}{\textcolor{special}{Public Health Reports. 2022;137(2):197-202}.} \\

2021 & Islam A, \textbf{McKee C}, Ghosh PK, Abedin J, Epstein JH, Daszak P, Luby SP, Khan SU,
Gurley ES. Seasonality of date palm sap feeding behavior by bats in Bangladesh. \href{https://doi.org/10.1007/s10393-021-01561-9}{\textcolor{special}{EcoHealth. 2021;18:359–71}.} \\

2021 & Zorrilla VO, Lozano ME, Espada LJ, Kosoy M, \textbf{McKee C}, Valdivia HO, Arevalo H, Troyes M, Stoops CA, Fisher ML, Vásquez GM. Comparison of sand fly trapping approaches for vector surveillance of \textit{Leishmania} and \textit{Bartonella} species in ecologically distinct, endemic regions of Peru. \href{https://doi.org/10.1371/journal.pntd.0009517}{\textcolor{special}{PLoS Neglected Tropical Diseases. 2021;15(7):e0009517}.} \\

2021 & \textbf{McKee CD}, Islam A, Luby SP, Salje H, Hudson PJ, Plowright RK, Gurley ES. The ecology of Nipah virus in Bangladesh: a nexus of land-use change and opportunistic feeding behavior in bats. \href{https://doi.org/10.3390/v13020169}{\textcolor{special}{Viruses. 2021;13(2):169}.} \\

2021 & \textbf{McKee C}, Bai Y, Webb C, Kosoy M. Bats are key hosts in the radiation of mammal-associated \textit{Bartonella} bacteria. \href{https://doi.org/10.1016/j.meegid.2021.104719}{\textcolor{special}{Infection, Genetics and Evolution. 2021;89:104719}.} \\

2020 & Goodrich I, \textbf{McKee C}, Kosoy M. \textit{Trypanosoma} (\textit{Herpetosoma}) diversity in rodents and lagomorphs of New Mexico with a focus on epizootological aspects of infection in Southern Plains woodrats (\textit{Neotoma micropus}). \href{https://dx.plos.org/10.1371/journal.pone.0244803}{\textcolor{special}{PLoS ONE. 2020;15(12):e0244803}.} \\

2020 & Rudolf I, Blazejova H, Mendel J, Strakova P, Sebesta O, Rettich F, Cabanova V, Miterpakova M, Betasova L, Pesko J, Barbusinova E, \textbf{McKee C}, Osikowicz L, Sikutova S, Hubalek Z, Kosoy M. 2020. \textit{Bartonella} species in medically important mosquitoes, Central Europe. \href{https://doi.org/10.1007/s00436-020-06732-1}{\textcolor{special}{Parasitology Research. 2020;119(8):2713-7}.} \\

2020 & Goodrich I, \textbf{McKee C}, Kosoy M. Longitudinal study of bacterial infectious agents in a community of small mammals in New Mexico. \href{https://doi.org/10.1089/vbz.2019.2550}{\textcolor{special}{Vector-Borne and Zoonotic Diseases. 2020;20(7):496-508}.} \\

2019 & \textbf{McKee CD}, Krawczyk AI, Sandor AD, Gorfol T, Foldvari M, Foldvari G, Dekeukeleire D, Haarsma A-J, Kosoy MY, Webb CT, Sprong H. Host phylogeny, geographic overlap, and roost sharing shape parasite communities in European bats. \href{https://doi.org/10.3389/fevo.2019.00069}{\textcolor{special}{Frontiers in Ecology and Evolution. 2019;7:69}.} \\

2018 & Bai Y, Osinubi MOV, Osikowicz L, \textbf{McKee C}, Vora NM, Rizzo MR, Recuenco S, Davis L, Niezgoda M, Ehimiyein AM, Kia GSN, Oyemakinde A, Adeniyi OS, Gbadegesin YH, Saliman OA, Ogunniyi A, Ogunkoya AB, Kosoy MY, Idanre Bat Festival Investigation Team. Human exposure to novel \textit{Bartonella} species from contact with fruit bats. \href{https://doi.org/10.3201/eid2412.181204}{\textcolor{special}{Emerging Infectious Diseases. 2018;24(12):2317-23}.} \\

2018 & Kellner A, Carver S, Scorza V, \textbf{McKee CD}, Lappin M, Crooks KR, VandeWoude S, Antolin MF. Transmission pathways and spillover of an erythrocytic bacterial pathogen from domestic cats to wild felids. \href{https://doi.org/10.1002/ece3.4451}{\textcolor{special}{Ecology and Evolution. 2018;8(19):9779-92}.} \\

2018 & \textbf{McKee CD}, Osikowicz LM, Schwedhelm TR, Maes SE, Enscore RE, Gage KL, Kosoy MY. Acquisition of \textit{Bartonella elizabethae} by experimentally exposed Oriental rat fleas (\textit{Xenopsylla cheopis}; Siphonaptera, Pulicidae) and excretion of bartonella DNA in flea feces. \href{https://doi.org/10.1093/jme/tjy085}{\textcolor{special}{Journal of Medical Entomology. 2018;55(5):1292-1298}.} \\

2018 & Kosoy M\textsuperscript{\dag}, \textbf{McKee C}\textsuperscript{\dag}, Albayrak L, Fofanov Y. Genotyping of \textit{Bartonella} bacteria and their animal hosts: current status and perspectives. \href{https://doi.org/10.1017/S0031182017001263}{\textcolor{special}{Parasitology. 2018;145(5):543-62}.} \\

2018 & Gorsich EE, \textbf{McKee CD}, Grear DA, Miller RS, Portacci K, Lindstrom T, Webb CT. Model-guided suggestions for targeted surveillance based on cattle shipments in the US. \href{https://doi.org/10.1016/j.prevetmed.2017.12.004}{\textcolor{special}{Preventive Veterinary Medicine. 2018;150:52-9}.} \\

2018 & \textbf{McKee CD}, Osikowicz LM, Schwedhelm TR, Bai Y, Castle KT, Kosoy MY. Survey of parasitic bacteria in bat bugs, Colorado. \href{https://doi.org/10.1093/jme/tjx155}{\textcolor{special}{Journal of Medical Entomology. 2018;55(1):237-41}.} \\

2017 & \textbf{McKee CD}, Kosoy MY, Bai Y, Osikowicz LM, Franka R, Gilbert AT, Boonmar S, Rupprecht CE, Peruski LF. Diversity and phylogenetic relationships among \textit{Bartonella} strains from Thai bats. \href{https://doi.org/10.1371/journal.pone.0181696}{\textcolor{special}{PLoS ONE. 2017;12(7):e0181696}.} \\

2017 & Urushadze L, Bai Y, Osikowicz L, \textbf{McKee C}, Sidamonidze K, Putkaradze D, Imnadze P, Kandaurov A, Kuzmin I, Kosoy M. Prevalence, diversity, and host associations of \textit{Bartonella} strains in bats from Georgia (Caucasus). \href{https://doi.org/10.1371/journal.pntd.0005428}{\textcolor{special}{PLoS Neglected Tropical Diseases. 2017;11(4):e0005428}.} \\

2017 & Bai Y, Urushadze L, Osikowicz L, \textbf{McKee C}, Kuzmin I, Kandaurov A, Babuadze G, Natradze I, Imnadze P, Kosoy M. Molecular survey of bacterial zoonotic agents in bats from the country of Georgia (Caucasus). \href{https://doi.org/10.1371/journal.pone.0171175}{\textcolor{special}{PLoS ONE. 2017;12(1):e0171175}.} \\

% FORCE PAGE BREAK
\end{tabular}
\section{Publications (continued)}
\begin{tabular}{>{\raggedright\arraybackslash}p{2cm}p{16cm}}

2016 & \textbf{McKee CD}, Hayman DTS, Kosoy MY, Webb CT. Phylogenetic and geographic patterns of bartonella host shifts among bat species. \href{https://doi.org/10.1016/j.meegid.2016.07.033}{\textcolor{special}{Infection, Genetics and Evolution. 2016;44:382-94}.} \\

2015 & Zinzow-Kramer WM, Horton BM, \textbf{McKee CD}, Michaud JM, Tharp GK, Thomas JW, Tuttle EM, Yi S, Maney DL. Genes located in a chromosomal inversion are correlated with territorial song in white-throated sparrows. \href{https://doi.org/10.1111/gbb.12252}{\textcolor{special}{Genes, Brain and Behavior. 2015;14(8):641-54}.} \\

2015 & Bai Y, Hayman DTS, \textbf{McKee CD}, Kosoy MY. Classification of \textit{Bartonella} strains associated with straw-colored fruit bats (\textit{Eidolon helvum}) across Africa using a multi-locus sequence typing platform. \href{https://doi.org/10.1371/journal.pntd.0003478}{\textcolor{special}{PLoS Neglected Tropical Diseases. 2015;9(1):e0003478}.} \\

\end{tabular}
\sectionspace

%------------------------------------------------
% OTHER CONTRIBUTIONS
%------------------------------------------------

\section{Other Contributions}
\begin{tabular}{>{\raggedright\arraybackslash}p{2cm}p{16cm}}
2021 & 2019 Novel Coronavirus Research Compendium (NCRC) $\cdot$ \href{https://ncrc.jhsph.edu/}{\textcolor{special}{https://ncrc.jhsph.edu/}.} \\

2015 & Owers K, \textbf{McKee C}, Hallman C, Portacci L, Miller R, Lindstrom T, Webb C. USAMM R Shiny Visualization $\cdot$ \href{https://usamm-gen-net.shinyapps.io/usamm-gen-net/}{\textcolor{special}{https://usamm-gen-net.shinyapps.io/usamm-gen-net/}.} \\
\end{tabular}
\sectionspace

%----------------------------------------------------------------------------------------
%	FELLOWSHIPS - GRANTS - AWARDS
%----------------------------------------------------------------------------------------

\section{Fellowships $\cdot$ Grants $\cdot$ Awards} 
\raggedright\textbf{Total received}: \$96,150\\
\sectionspace
\begin{tabular}{>{\raggedright\arraybackslash}p{2cm}p{16cm}}
2018 & \textbf{Colorado State University} Vice President for Research Fellowship $\cdot$ \$4,000\\
2015–2017 & \textbf{CDC Division of Vector-Borne Diseases} Regular Fellowship $\cdot$ \$82,000\\
2014 \& 2017 & \textbf{Colorado State University} Department of Biology Travel Awards $\cdot$ \$2,000\\
2015 & \textbf{Colorado State University} Sharon E and David E Kabes Scholarship $\cdot$ \$1,150\\
2014 & \textbf{Colorado State University} Graduate Degree Program in Ecology Small Research Grant $\cdot$ \$2,000\\
2013 & \textbf{Colorado State University} Graduate Fellowship $\cdot$ \$1,000\\
2011 & \textbf{University of Pittsburgh} Phi Beta Kappa\\
2007 & \textbf{University of Pittsburgh} Honors College Scholarship $\cdot$ \$4,000\\
\end{tabular}
\sectionspace

%------------------------------------------------
% TEACHING 
%------------------------------------------------

\section{Teaching}
\begin{tabular}{>{\raggedright\arraybackslash}p{2cm}p{16cm}}
2022 & \textbf{Topics in Infectious Disease Epidemiology} Teaching Assistant $\cdot$ Johns Hopkins University\\
2021 \& 2022 & \textbf{The One Health Approach to Epidemiology and Global Public Health} Teaching Assistant $\cdot$ Johns Hopkins University\\
2020 \& 2021 & \textbf{Epidemiology of Infectious Diseases} Teaching Assistant $\cdot$ Johns Hopkins University\\
2018 \& 2019 & \textbf{Molecular and General Genetics} Teaching Assistant $\cdot$ Colorado State University\\
2018 & \textbf{Introduction to Evolution} Teaching Assistant \& Guest Lecturer $\cdot$ Colorado State University\\
2018 & \textbf{Principles of Systematic Zoology} Guest Lecturer $\cdot$ Colorado State University\\
2017 & \textbf{Ecology} Teaching Assistant $\cdot$ Colorado State University\\
2017 & \textbf{Tropical Ecology and Evolution} Guest Lecturer $\cdot$ Colorado State University\\
2016 & \textbf{Ecology of Infectious Diseases} Guest Lecturer $\cdot$ Colorado State University\\
2014 \& 2015 & \textbf{Ornithology} Teaching Assistant \& Guest Lecturer $\cdot$ Colorado State University\\
2014 & \textbf{Molecular and General Genetics} Teaching Assistant \& Guest Lecturer $\cdot$ Colorado State University\\
2013 & \textbf{Attributes of Living Systems} Teaching Assistant $\cdot$ Colorado State University\\
\end{tabular}
\sectionspace

%------------------------------------------------
% STUDENTS
%------------------------------------------------

\section{Student Mentoring}
\begin{tabular}{>{\raggedright\arraybackslash}p{2cm}p{16cm}}
2018 & \textbf{Paul Lazarus} Undergraduate Research Assistant $\cdot$ Colorado State University\\
2016–2017 & \textbf{Alexis Cannella} Undergraduate Research Assistant $\cdot$ Colorado State University\\
2014–2015 & \textbf{Jake Doyle} Undergraduate Research Assistant $\cdot$ Colorado State University\\
2014–2015 & \textbf{Madison Leming} Undergraduate Research Assistant $\cdot$ Colorado State University\\
\end{tabular}
\sectionspace

%------------------------------------------------
% PRESENTATIONS
%------------------------------------------------

\section{Presentations}
\begin{tabular}{>{\raggedright\arraybackslash}p{2cm}p{16cm}}

2019 & \textbf{Evolution} Providence, RI $\cdot$ Contributed talk: Comprehensive time tree analysis identifies bats as key to the radiation of mammal-associated \textit{Bartonella} bacteria\\

2019 & \textbf{Ecology and Evolution of Infectious Disease Conference} Princeton, NJ $\cdot$ \textcolor{special}{Best poster award} $\cdot$ Poster: Manipulating vector transmission reveals local processes in bacterial communities of bats\\

2019 & \textbf{Front Range Student Ecology Symposium} Fort Collins, CO $\cdot$ \textcolor{special}{Best poster award} $\cdot$ Poster: Dispersal of hosts and vectors predicts bacterial community structure across an island chain\\

2018 & \textbf{Ecological Society of America Annual Meeting} New Orleans, LA $\cdot$ Contributed talk: Timing the diversification of a mammal parasite, \textit{Bartonella}\\

2018 & \textbf{Front Range Student Ecology Symposium} Fort Collins, CO $\cdot$ Contributed talk: Timing the diversification of a mammal parasite, \textit{Bartonella}\\

2018 & \textbf{Vice President for Research Three Minute Challenge} Fort Collins, CO $\cdot$ Invited talk: Bad roommates? Inferring bacterial interactions in coinfected individuals\\

2017 & \textbf{Colorado State University Graduate Student Showcase} Fort Collins, CO $\cdot$ Poster: Timing the diversification of a mammal parasite, \textit{Bartonella}\\

2017 & \textbf{Ecological Society of America Annual Meeting} Portland, OR $\cdot$ Contributed talk: Long-term monitoring of \textit{Bartonella} spp. bacteria in a captive colony of fruit bats and experimental evidence of bat flies as vectors of bartonella\\

2017 & \textbf{International Symposium on Infectious Diseases of Bats} Fort Collins, CO $\cdot$ Contributed talk: Long-term monitoring of \textit{Bartonella} spp. bacteria in a captive colony of fruit bats and experimental evidence of bat flies as vectors of bartonella\\

2017 & \textbf{Ecology and Evolution of Infectious Disease Conference} Santa Barbara, CA $\cdot$ Poster: Linking patterns of bacterial parasite diversity across host and vector communities\\

2017 & \textbf{Front Range Student Ecology Symposium} Fort Collins, CO $\cdot$ Contributed talk: Host phylogenetic distance and ectoparasite overlap predict \textit{Bartonella} sharing in European bats\\

2015 & \textbf{Ecological Society of America Annual Meeting} Baltimore, MD $\cdot$ Contributed talk: Phylogeography of \textit{Bartonella} bacteria in \textit{Eidolon} spp. fruit bats across Africa\\

2015 & \textbf{Ecology and Evolution of Infectious Disease Conference} Athens, GA $\cdot$ Poster: Phylogeography of \textit{Bartonella} bacteria in \textit{Eidolon} spp. fruit bats across Africa\\
	
2015 & \textbf{Front Range Student Ecology Symposium} Fort Collins, CO $\cdot$ Contributed talk: Phylogeography of \textit{Bartonella} in \textit{Eidolon} fruit bats across Africa\\

2014 & \textbf{Ecological Society of America Annual Meeting} Sacramento, CA $\cdot$ Poster: Large-scale patterns of \textit{Bartonella} prevalence and diversity in African fruit bats\\

2014 & \textbf{Ecology and Evolution of Infectious Disease Conference} Fort Collins, CO $\cdot$ Poster: Large-scale patterns of \textit{Bartonella} prevalence and diversity in African fruit bats\\

2014 & \textbf{Front Range Student Ecology Symposium} Fort Collins, CO $\cdot$ Poster: A novel method of isolating multi-locus sequence data for characterizing \textit{Bartonella} diversity in bats\\

\end{tabular}
\sectionspace

%------------------------------------------------
% PROFESSIONAL SERVICE
%------------------------------------------------

\section{Professional Service} 
\textbf{Peer review}: \textcolor{special}{Acta Chiropterologica},
	\textcolor{special}{Acta Tropica},
	\textcolor{special}{BMJ},
	\textcolor{special}{Current Zoology},
	\textcolor{special}{Frontiers in Microbiology},
	\textcolor{special}{Frontiers in Veterinary Science},
	\textcolor{special}{Infection, Genetics and Evolution},
	\textcolor{special}{Journal of Medical Entomology},
	\textcolor{special}{Journal of Zoological Systematics and Evolutionary Research},
	\textcolor{special}{Microbial Ecology},
	\textcolor{special}{Microbial Pathogenesis},
	\textcolor{special}{Pathogens and Global Health},
	\textcolor{special}{PLOS Neglected Tropical Diseases},
	\textcolor{special}{PLOS ONE},
	\textcolor{special}{The European Zoological Journal},
	\textcolor{special}{The Lancet Planetary Health},
	\textcolor{special}{The Science of Nature},
	\textcolor{special}{Scientific Reports},
	\textcolor{special}{Systematic Biology},
	\textcolor{special}{Tropical Biomedicine},
	\textcolor{special}{Veterinary Microbiology},
	\textcolor{special}{Viruses},
 	\textcolor{special}{Zoonoses and Public Health}\\
\sectionspace
\begin{tabular}{>{\raggedright\arraybackslash}p{2cm}p{16cm}}
2021 & \textbf{American Society of Tropical Medicine and Hygiene} Member\\
2019 \& 2021 & \textbf{Society for the Study of Evolution} Member\\
2019 & \textbf{American Society of Naturalists} Member\\
2014–2019 & \textbf{Ecological Society of America} Member\\
2014–2019 & \textbf{Front Range Student Ecology Symposium} Volunteer\\
2015 & \textbf{Front Range Student Ecology Symposium} Vice President\\
2014 & \textbf{Ecological Society of America} Volunteer\\
\end{tabular}
\sectionspace

%------------------------------------------------
% COMMUNITY ENGAGEMENT
%------------------------------------------------

\section{Community Engagement} 
\begin{tabular}{>{\raggedright\arraybackslash}p{2cm}p{16cm}}
2021–2022 & \textbf{Maryland Science Olympiad} Baltimore, MD $\cdot$ Disease Detectives Exam Writer\\
2018 \& 2019 & \textbf{Rocky Mountain Regional Middle School Science Bowl} Loveland, CO $\cdot$ Moderator\\
2018 & \textbf{National Science Olympiad} Fort Collins, CO $\cdot$ Workshop on Disease Ecology\\
2015 & \textbf{O'Dea Elementary School Science Fair} Fort Collins, CO $\cdot$ Judge\\
\end{tabular}
\sectionspace

%------------------------------------------------
% REFERENCES
%------------------------------------------------

%\section{References}
%\textbf{Dr Colleen Webb} Professor $\cdot$ Department of Biology $\cdot$ Colorado State University\\
%\href{mailto:colleen.webb@colostate.edu}{\Letter~colleen.webb@colostate.edu}\\
%\href{tel:+19704916723}{\Mobilefone~+1 970 491 6723}\\
%
%\sectionspace
%\textbf{Dr Michael Kosoy} Research Biologist $\cdot$ KB ONE Health\\
%\href{mailto:kosoymichael@gmail.com}{\Letter~kosoymichael@gmail.com}\\
%\href{tel:+19702663522}{\Mobilefone~+1 970 266 3522}\\
%
%\sectionspace
%\textbf{Dr David TS Hayman} Associate Professor $\cdot$ Institute of Veterinary, Animal and Biomedical Sciences $\cdot$ Massey University\\
%\href{mailto:d.t.s.hayman@massey.ac.nz}{\Letter~d.t.s.hayman@massey.ac.nz}\\
%\href{tel:+640279098802}{\Mobilefone~+64 0 27 909 8802}\\

\end{document}